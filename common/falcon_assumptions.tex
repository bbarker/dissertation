% I put this in to a separate file because formatting the table in different ways
% is difficult; it may even be better to have multiple versions of this table
% for different documents, but hopefully we can avoid such code duplication.

%Internal part of the table:

\begin{enumerate}
% This is really not related to GPR rules: 
%\ifthenelse{\boolean{thesisStyle}}{\item} {} \label{asm:mm}
%Fluxes in general strive to operate near the $V_{max}$ of the
%reaction, which is proportional to enzyme complex abundance.
\item \label{asm:expcorr}
Expression values are highly correlated with the copy numbers of their
corresponding peptide isoforms.
\item \label{asm:isozyme} 
Protein isoforms contributing to isozymes are considered part of the
same enzyme complex.
\item \label{asm:hierarchy}
Any enzyme complex can be described as a hierarchical subset of
(possibly redundant) subcomplexes; redundant subcomplexes, as
elaborated in (\ref{asm:nostoich}), are not currently modeled.
\item \label{asm:nostoich} 
Assume one copy of peptide per complex; exact isoform stoichiometry
is not considered.
\item \label{asm:sharing} 
With the exception of complexes having identical rules (i.e. the same
complex listed for different reactions), each copy of a peptide
is available for all complexes in the model.
\item \label{asm:active_site}
There is only one active site per enzyme complex.
\item \label{asm:enzyme_sensitivity} 
We assume that different pathways have similar flux sensitivities
with respect to their enzyme abundances.
\item \label{asm:holo} 
If a particular subcomplex can be catalyzed by A and it can also be
catalyzed by A and B (e.g. B acts as a regulatory unit, as in
holoenzymes), this just simplifies to A once expression values are
substituted in. Similarly, allosteric regulation is not
modeled. Relatedly, there are no NOT operations in GPR rules (just ANDs
and ORs).
\item \label{asm:chap} 
Enzyme complexes form without the assistance of protein chaperones and
formation is not coupled to other reactions.  
\item \label{asm:rate} 
Rate of formation and degradation of complexes doesn't play a role,
since we assume steady-state. 
\end{enumerate}
