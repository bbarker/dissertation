\newcommand{\eeFBAfigOneCap}{%
More positive differential epistases under environmental
perturbations. (\textbf{A}) Heat maps describe the global dynamics of
differential epistasis from abundant-glucose medium to ethanol (left
panel) and glycerol (right panel) conditions. Only gene pairs with
$\left|\D\epsilon\right| \geq 0.01$ in either condition are included
in the figure. Different colors represent differential epistasis
values as indicated by the color bar at the bottom. The differential
epistasis values are assigned to be 0.1 (or -0.1) in the heat-maps
when it is greater than 0.1 (or less than -0.1). It is noteworthy to
point out that the epistasis patterns are indeed very different
between the two conditions (\Fig~\ref{fig:eef2}A). (\textbf{B})
Percentage of positive and negative differential epistases under
ethanol and glycerol conditions. (\textbf{C}) Ratio of positive to
negative differential epistases
in each simulated condition. The result from a high-throughput
experiment is also shown. The letters A-P represent acetaldehyde,
acetate, adenosine 3',5'-bisphosphate, adenosyl methionine, adenosine,
alanine, allantoin, arginine, ethanol, glutamate, glutamine, glycerol,
low glucose, phosphate, trehalose, and xanthosine, respectively.
}

\newcommand{\eeFBAfigTwoCap}{%
Epistasis dynamics between environmental perturbations. (\textbf{A}) Number of
gene pairs with various epistatic relationships between ethanol and
glycerol growth conditions. (\textbf{B}) The distribution for the percentages
of gene pairs with similar epistasis relation between any 2 of 16
conditions. The frequency is derived from the 120 pairs of
environmental conditions simulated in this study.
}

\newcommand{\eeFBAfigThreeCap}{%
The global distribution of epistatic relations under simulated
conditions. (\textbf{A}) Distribution for the number of conditions in which
each epistatic interaction exists. Note that $\approx$~28\% of
epistatic relations are extremely stable (the very right bar) and
$\approx$~24\% are extremely dynamic (the very left bar). (\textbf{B}) Fraction
of three types of epistatic relations in each of the 16 environmental
perturbations, as indicated by the color bar to the right. The numbers
in the brackets represent the number of conditions in which each
epistatic interaction exists, as indicated in (\textbf{A}). The letters A-P
represent the simulated conditions as indicated in Figure 1.
}

\newcommand{\eeFBAfigFourCap}{%
Network properties for the extremely stable and extremely dynamic
epistatic interactions. (\textbf{A}) Degree distribution for genes in two
epistatic interaction networks. The networks have nodes that
correspond to genes and edges that correspond to epistatic
interactions. (\textbf{B}) Three network parameters (the definition of which
are shown in Methods) for two epistatic interaction networks.
}

\newcommand{\eeFBAfigFiveCap}{%
Co-evolution between genes with epistasis. (\textbf{A}) Average evolutionary
rate differences between gene pairs with FBA-predicted epistasis
(green), extremely dynamic epistasis (blue) and extremely stable
epistasis (red) are highlighted by three arrows, respectively. The
random simulations with the same number of gene pairs as each of the
three groups were repeated 10,000 times and the frequency
distributions are shown (marked by the same colors as the
corresponding arrows, respectively). (\textbf{B}) The evolutionary rates for
genes that are involved in extremely stable and extremely dynamic
epistasis, respectively. The error bars represent standard errors.
}

\newcommand{\eeFBAfigSOneCap}{%
More positive differential epistases under environmental perturbations
for different thresholds of differential epistasis
($\left|\D\epsilon\right| \geq 0.001$, \textbf{A}) and
($\left|\D\epsilon\right| \geq 0.05$, \textbf{B}). Ratio of positive
to negative differential epistases in each simulated condition are
shown. The letters A-P represent acetaldehyde, acetate, adenosine
3',5'-bisphosphate, adenosyl methionine, adenosine, alanine,
allantoin, arginine, ethanol, glutamate, glutamine, glycerol, low
glucose, phosphate, trehalose, and xanthosine, respectively.
}

\newcommand{\eeFBAfigSTwoCap}{%
Analogous to \Fig~\ref{fig:eef1}B-C, but using a maximum growth rate for each
condition, where the maximum is constrained to be no higher than the
high-glucose growth rate. (\textbf{A}) Percentage of positive and
negative differential epistases under ethanol and glycerol
conditions. (\textbf{B}) Ratio of positive to negative differential
epistases in each simulated condition. The result from a
high-throughput experiment is also shown. The letters A-P represent
acetaldehyde, acetate, adenosine 3',5'-bisphosphate, adenosyl
methionine, adenosine, alanine, allantoin, arginine, ethanol,
glutamate, glutamine, glycerol, low glucose, phosphate, trehalose, and
xanthosine, respectively. Note that in (B), low glucose has the same
growth rate as high-glucose, but has different epistatic interactions
since we still use the high-oxygen uptake level associated with the
low glucose condition.
}

\newcommand{\eeFBAfigSThreeCap}{%
Epistasis dynamics between environmental perturbations under different
epistasis definition. (\textbf{A}) Number of gene pairs with various
epistatic relationships between ethanol and glycerol growth conditions
under a lower ($\left|\epsilon\right| \geq 0.001$) and a higher
($\left|\epsilon\right| \geq 0.05$) epistasis threshold. (\textbf{B})
The distribution for the percentages of gene pairs with similar
epistasis relations between any 2 of 16 conditions under a lower
($\left|\epsilon\right| \geq 0.001$) and a higher
($\left|\epsilon\right| \geq 0.05$) epistasis threshold.
}

\newcommand{\eeFBAfigSFourCap}{%
Analogous to \Fig~\ref{fig:eef3}A, but using a maximum growth rate for each
condition, where the maximum is constrained to be no higher than the
high-glucose growth rate. Distribution for the number of conditions in
which each epistatic interaction exists. Note that $\approx$~26\% of
epistatic relations are extremely stable (the very right bar) and
$\approx$~19\% are extremely dynamic (the very left bar).
}

\newcommand{\eeFBATabSOneCap}{% 
Wild-type growth rates used in the maximal growth rate simulations
used for \Fig s \ref{fig:eefS2} and \ref{fig:eefS4}.
}

\newcommand{\eeFBATabSTwoCap}{% 
Condition-specific epistates and sign-epistases prevalence in the
iMM904 yeast model.
}
\newcommand{\eeFBATabSThreeCap}{% 
GO term enrichment analysis results for differential epistasis in
transition to ethanol.
}

\newcommand{\eeFBATabSFourCap}{% 
Properties of simulated systems that correlate with the ratio of
positive to negative differential epistases.
}

\newcommand{\eeFBATabSFiveCap}{% 
List of epistatic interactions for the extremely stable, dynamic, and
intermediate epistasis networks.
}

\newcommand{\eeFBATabSSixCap}{% 
Table of network parameters for stable, dynamic, and intermediate
epistasis.
}
