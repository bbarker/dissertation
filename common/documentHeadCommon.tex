% Define block styls
\tikzstyle{line} = [draw, very thick, color=black!50, -latex']
\tikzstyle{cloud} = [draw, ellipse,fill=red!20, node distance=2em]
\tikzstyle{decision} = [diamond, draw, fill=blue!20, align=center,
    text badly centered, node distance=6em, inner sep=0pt]
\tikzstyle{block} = [rectangle, draw, %fill=blue!20,
    text centered, align=center, node distance=4em]
\tikzstyle{iogram} = [trapezium, draw, fill=black!10, %fill=pink!20,
    trapezium left angle=70, trapezium right angle=-70,
    text centered, align=center, node distance=5em]
\tikzset{
  onslide/.code args={<#1>#2}{\only<#1>{\pgfkeysalso{#2}}}, 
}

% Some global commands for alternatives

\DeclareRobustCommand\suppOrApp{%
  \ifthenelse{\boolean{thesisStyle}}%
    {Appendix}
    {Supplementary}
}
\DeclareRobustCommand\Fig{%
  \ifthenelse{\boolean{thesisStyle}}%
    {Figure}%
    {Fig.}%
}



% \ifthenelse{\boolean{thesisStyle}}
% { 
% % \renewcommand{\caption}[1]{\singlespacing\hangcaption{#1}\normalspacing}
%   \renewcommand{\topfraction}{0.85}
%   \renewcommand{\textfraction}{0.1}
% %
% % This seems to cause a problem with floats and excess vertical space:
% %
%   \renewcommand{\floatpagefraction}{0.75}
% }
% {}

\ifthenelse{\boolean{thesisStyle}}{%
  \floatstyle{plaintop}
  \restylefloat{table}
}
{}


% \epstopdfDeclareGraphicsRule{.eps}{pdf}{.pdf}{%
%   epstool --copy --bbox #1 --output epstool_tmp.eps;
%   epstopdf --hires epstool_tmp.eps --outfile \OutputFile;
%   rm epstool_tmp.eps
% }


\def\D{\mathrm{d}}

%\newcommand{\ANDw}{\textnormal{AND}}
%\newcommand{\ORw}{\textnormal{OR}}
\newcommand{\ANDw}{\land}
\newcommand{\ORw}{\lor}


%%%%%%%%%%%%%%%%%%%%%%%%%%%% Introduction %%%%%%%%%%%%%%%%%%%%%%%%%%%%
\newcommand{\introSameGeneCredit}{%
\ifthenelse{\boolean{thesisStyle}}{%
  \footnote{This chapter is taken from material in \citealt{Shestov2013a}.
Brandon Barker is the primary author of all material found herein.}%
}%
{}%
}


%% Dynamic Epistasis for Different Alleles of the Same Gene %%

\newcommand{\epiSameGeneCredit}{%
\ifthenelse{\boolean{thesisStyle}}{%
  \footnote{This chapter is published as \citet{Xu2012}.
Brandon Barker and Lin Xu contributed equally to this work.
It is additionally available in \citet[chapter 4]{Xu2012a}.}%
}%
{}%
}

\newcommand{\epiSameGeneAbstract}{%
Epistasis refers to the phenomenon in which phenotypic consequences
caused by mutation of one gene depend on one or more mutations at
another gene. Epistasis is critical for understanding many genetic and
evolutionary processes, including pathway organization, evolution of
sexual reproduction, mutational load, ploidy, genomic complexity,
speciation, and the origin of life. Nevertheless, current
understandings for the genome-wide distribution of epistasis are
mostly inferred from interactions among one mutant type per gene,
whereas how epistatic interaction partners change dynamically for
different mutant alleles of the same gene is largely unknown. Here we
address this issue by combining predictions from flux balance analysis
and data from a recently published high-throughput experiment. Our
results show that different alleles can epistatically interact with
very different gene sets. Furthermore, between two random mutant
alleles of the same gene, the chance for the allele with more severe
mutational consequence to develop a higher percentage of negative
epistasis than the other allele is 50-70\% in eukaryotic organisms,
but only 20-30\% in bacteria and archaea. We developed a population
genetics model that predicts that the observed distribution for the
sign of epistasis can speed up the process of purging deleterious
mutations in eukaryotic organisms. Our results indicate that epistasis
among genes can be dynamically rewired at the genome level, and call
on future efforts to revisit theories that can integrate epistatic
dynamics among genes in biological systems\epiSameGeneCredit.
}


%%%%%%%%%%%%%%%%%%%%%%%%%%%%% FALCON %%%%%%%%%%%%%%%%%%%%%%%%%%%%%

\newcommand{\falconAbstractMotivation}{%
A major theme in constraint-based modeling is unifying 
experimental data, such as biochemical information about the reactions
that can occur in a system or the composition and localization of enzyme
complexes, with high-throughput data including expression data,
metabolomics, or DNA sequencing. The desired result is to increase
 predictive capability resulting in improved understanding of metabolism.
 The approach typically employed when only gene (or protein) intensities
are available is the creation of tissue-specific models, which reduces
the available reactions in an organism model, and does not provide an
objective function for the estimation of fluxes, which is an important
limitation in many modeling applications.
}

\newcommand{\falconAbstractResults}{%
We develop a method, flux assignment with LAD (least absolute
deviation) convex objectives and normalization (FALCON),
 that employs metabolic network reconstructions along with expression
data to estimate fluxes. In order to use such a method, accurate
measures of enzyme complex abundance are needed, so we first
present a new algorithm that addresses quantification of complex
abundance. Our extensions to prior techniques include the
capability to work with large models and significantly improved
run-time performance even for smaller models, an improved analysis of
enzyme complex formation logic, the ability to handle very large enzyme
complex rules that may incorporate multiple isoforms, and depending on
the model constraints, either maintained or significantly improved
correlation with experimentally measured fluxes.
}

\newcommand{\falconAbstractAvail}{%
FALCON has been implemented in MATLAB and ATS, and can be downloaded
from: \url{https://github.com/bbarker/FALCON}. ATS is not required to
compile the software, as intermediate C source code is available, and
binaries are provided for Linux x86-64 systems. FALCON requires use
of the COBRA Toolbox, also implemented in MATLAB.
}

\captionsetup{labelfont=bf}


% Take care of potentially defined variables:

% \newcommand[1]{\identifndef}{
%   \if\isdef\csname{#1}
%     {}
%   \else
%     \newcommand{\{#1}}[1]{\{#1}}
%   \fi
% }

% \identifndef{processtable}