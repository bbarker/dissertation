
To our knowledge, existing literature has only dealt with the
simulation of only deleterious mutants rather than beneficial mutants
in the constraint-based modeling literature, which is chiefly due to
existing studies optimizing the fitness function, which leaves no room
for improvement. In this study, we develop a constraint-based
approach that can simulate beneficial, neutral, and deleterious
mutations.  We show that this simulation technique can be useful for
understanding adaptive trajectories, for which we develop an
analytical software library.  Our mechanistic model can reproduce the
distribution of epistases between beneficial mutations that was
observed in a data set and a population genetic model fit to the same
data set, showing that our model behaves appropriately in this conext
and may be a useful tool for further evolutionary analyses. Finally,
in experimental data sets and in our simulations, slightly beneficial
mutations are much more likely to have positive (synergistic)
epistasis with other beneficial mutations, making their likelihood of
becoming fixed higher than would be expected without considering
epistatic effects.


\section{Introduction}

\subsection{Adaptive Mutations}

Biologists have long wondered the extent to which evolution occurs due
to nearly neutral and slighly deleterious mutations, or adaptive
mutations, or more complex situations involving these types of
mutations as well as their epistatic interactions (cite the work of
Kimura etc.).

The occurence of beneficial mutations and how they affect adaptation
is currently an area of active interest in evolutionary biology
\cite{Chou2011} \cite{Weinreich2006}. Although much focus in the past
has been placed on deleterious mutations because of their prevalence
in nature and disease, it is ultimately beneficial mutations that
are responsible for adaptive evolution.
 
\subsection{Adaptive Mutations and Epistasis}



\subsection{The Need for a New Modeling Framework}
Clearly, using FBA with the growth objective alone is not enough--we
only ever get the optimum for our fitness objective.  A way to
circumvent this issue is to associate a feature of the system
(e.g. flux into biomass) as the fitness while optimizing some other
objective. This latter mechanism is not generally used as most models
tend to be rather under-constrained as is. However, as we've seen, the
FALCON method \hl{cite} and other fitting methods like MoMA provide a
way to use high-throughput data to introduce many additional
constraints to the system.

Having a systems tool that can work with models of particular organisms
will not only add another tool in the computational evolution and
population genetics arsenal, but also in applied fields such as
evolutionary engineering of microbial engineering, and understanding
which gene mutant combinations which may be most advantageous for
a cancer cell population.

\section{Results}

\subsection{Adaptive mutations with objective weights}

Aside from the problem of separating the fitness function from the 
optimization objective function, there is the issue of combining traditional flux
restriction mutations, which are known as \emph{hard constraints}, which may
result in an unsolvable system --- an almost certainly undesirable
effect of this mutation modeling formalism. Instead, it would be
better if mutations could be modeled as \emph{soft
  constraints}. Concretely, whereas hard constraints are enacted in
the actual constraints of the optimization problem, soft constraints
merely change the objective. This means that multiple soft constaints
combined together under some mutational model would be compatible in
the sense that they wouldn't unexpectedly result in an unsolvable
system. FALCON provides two possible avenues for soft constraints:
expression level and expression variation. \emph{However, it is not
  clear yet what expression variation really means, so further
  investigation is necessary.}


\section{Discussion}

The questions are often
difficult or impossible to assess experimentally due to limited
resources.  In genome-scale models, to our knowledge, only microbial
epistasis has so far been studied for all enzymes (often referred to
as genome-scale in this context). This is due to several factors.

One issue is that these computations can still take a significant
amount of time, and the increase in model size of Human Recon 2 over
Yeast can cause even a relatively simple FBA run to go up by an order
of magnitude.  This problem is compounded by the increase in the
number of genes in the human model, since computing epistasis consumes
space and time as $O(n^2)$ where $n$ is the number of genes in the
model. More important than this issue, which might be overcome with
enough computational resources, is the issue of an objective
function. It has been shown numerous times that FBA with a biomass
objective can be a reasonable approximation to what a microbe is
trying to achieve metabolically
~\cite{Schuetz2012}~\cite{Fong2004}~\cite{Varma1994} . While Recon 2
is equipped with a ``generalized biomass reaction'', it is not clear
what the meaning of this is, and it certainly seems unlikely to
estimate the metabolism even of fast-growing cancer cells \hl{(cite
  Locasale?)}. We propose FALCON as a method to get around this issue
for non-microbial models.

Another advantage of FALCON is that it allows one to directly probe
mutations that are represented as gene expresion perturbations. A
decreased level of gene expression may also be metabolically
equivalent to the effect of a missense mutation, for example. This
allows a different sampling strategy than before; for instance, we
could observe how uniform expression restriction compares to uniform
flux restriction~\cite{Xu2012}. Assuming an accurate model of
enzyme-complex expression measurement, the former should be the more
realistic model.

A limitation is that we have only considered metabolic genes and their
effect on steady-state metabolism. While in principle a similar method
could be applied to whole cell models \hl{cite myco and e.coli}, the
computation time would not be feasible to the screening for beneficial
muatations, nor of exploring them combinatorally, as the time needed
for a single mutant takes at least a day even in the smallest
bacterial model \hl{cite}. Future isnights into improving the
efficiency of whole cell models, or making a compromise on which
systems are simulated, may improve these efforts.

\section{Methods}
\hl{it is empty}

\section{TODO}

\hl{need more background}


\hl{include weighted moma subsection to introduce need for FBA
in conjunction with FALCON; tie in this multiobjective function
to the multiobjective optimality found by Shuetz}

\hl{add images of Lin's experimental data}

\hl{actually try to use FALCON for GxG interactions, discuss caveat
that it is possible that influencing one enzymatic gene's abundance
could affect another enzymatic gene's abundance, and we can't directly
take this into account in our model}

\hl{discuss C software I developed for analysing adaptive trajectories;
need to update to work with FALCON}


\hl{briefly discuss FBA weight possibility}


\hl{Since beneficial mutations are relatively rare, and since
combining multiple mutants in the lab is increasingly difficult for
the more mutations that are to be combined, an \textit{in silico}
analysis can shed light on what may be expected in adaptive
evolution.}

