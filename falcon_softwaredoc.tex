%
% TODO: use matlab-prettifier for matlab code: https://bitbucket.org/Jubobs/matlab-prettifier
%

\section{Introduction}

In order to develop software that is easy to use and shares a popular API,
our software is largely implemented in MATLAB using the COBRA
Toolbox \citep{Hyduke2011}. The first algorithm in the technique
has been implemented in ATS, a C-code generating language with rich
safety features, in order to attain good time-efficiency without sacrificing
safety and stability \citep{ATStypes03}. The only expensive operations in the
MATLAB code are external calls to a linear program (LP) solver,
so the efficiency in this case depends on the solver and the metabolic
model in a complicated manner due to diverse algorithms and
implementations as well as sensitivity to the model parameters
\citep{Mittelmann2013,Todd2002}. We discuss performance statistics for
several models below using a state-of-the-art solver (Table FalcPerf; 
\ref{ssec:lpsettings}). Before comparing this approach to related methods, we
describe our rationale and algorithms.

\section{Installing FALCON}

\hl {It is mostly empty.}

FALCON can be compiled without installing ATS since all ATS code is
translated to C and included in the package. However, ATS is quite
easy to install and users may wish to use some of the library
functions of FALCON written in ATS --- notably the methods for dealing
with GPR rules. To install ATS, directions can be found on the ATS2
wiki \citep{ATSWikiInstall}.

\section{Running FALCON}

There are several ways to run FALCON. The script that implements
Algorithm~\ref{alg:FALCON} is \textit{falcon.m}; this script takes
many optional values, and the function signature is highly likely
to change in the near future. For basic use, we suggest using
the wrapper script \textit{runFalcon.m}, which assumes several
parameter values. This wrapper could be easily changed to use
different parameter defaults, or to call \textit{falconMulti.m},
itself a wrapper with a similar function signature for
\textit{falcon.m} that calculates the mean and standard deviation of
fluxes across multiple calls to \textit{falcon.m}.

\section{Code excerpt for minimum disjunction algorithm}
\label{sec:code}

\hl{Need to update code here; try using atslistings.sty for code 
highlights.}

The following code is written in ATS, a type-safe language including
syntax similar to SML while having direct access to C types
\citep{ATStypes03}. \texttt{GREXP} describes a datatype that is used
for storing the parse trees of Boolean rules (without negation).  A
\texttt{GREXP} can then be manipulated by the function \texttt{toCNF}
to be converted to a conjunctive normal form. We make use of the
reduction rule mentioned previously by calling the \texttt{minConj}
function. The \texttt{conjunctivize} function is a helper function to
deal with different structures for \texttt{GRconj} and \texttt{GRdisj}
(one is set based, one is parse-tree based; note this described in the
data(view)type definition). We note that this is a recursive,
functional implementation of Algorithm~\ref{alg:ReductionToCNF}, which
seems more straightforward than a procedural implementation.

\begin{verbatim}
dataviewtype GREXP = 
  | GRgenes of genes
  | GRconj of genes
  | GRconj of (GREXP,GREXP)
  | GRdisj of genes
  | GRdisj of (GREXP,GREXP)

extern
fun toCNF (bexp: GREXP, emap: &gDMap): GREXP

implement
toCNF (bexp, emap): GREXP = let     
  val LR:GREXP = (case+ bexp of 
    | ~GRconj(ex1,ex2) => GRconj (toCNF(ex1,emap),toCNF(ex2,emap))
    | ~GRdisj(ex1,ex2) => GRdisj (toCNF(ex1,emap),toCNF(ex2,emap))   
    | GR => GR):GREXP
  in (case+ LR of  
    | ~GRconj(ex1,ex2) => minConj(GRconj(ex1,ex2),emap) 
    | ~GRdisj(ex1,ex2) => (case+ (ex1,ex2) of
      // Handle disjunctive leaf cases:         
      | (~GRdisj(lx), ~GRgenes(g)) => GRdisj (lx + g) 
      | (~GRgenes(g), ~GRdisj(rx)) => GRdisj (rx + g)
      | (~GRdisj(lx), ~GRdisj(rx)) => GRdisj (lx + rx)
      | (~GRgenes(g1), ~GRgenes(g2)) => GRdisj (g1 + g2)

      // Distribute OR over ANDs:
      | (~GRconj(x1,x2), ~GRconj(g)) => conj1(x1,x2,g,emap) 
      | (~GRconj(g), ~GRconj(x1,x2)) => conj1(x1,x2,g,emap) 
      | (~GRconj(g1), ~GRconj(g2)) => conj2(g1,g2,emap)
      | (~GRconj(lx1,lx2), ~GRconj(rx1,rx2)) => let
        val lx1c = GREXP_copy(lx1)
        val lx2c = GREXP_copy(lx2) 
        val rx1c = GREXP_copy(rx1)
        val rx2c = GREXP_copy(rx2) 
        in GRconj(GRconj(GRconj(toCNF(GRdisj(lx1,rx1),emap), 
          toCNF(GRdisj (lx2, rx1c ),emap)),
          toCNF(GRdisj (lx1c, rx2),emap)), 
          toCNF(GRdisj (lx2c, rx2c),emap)) 
        end

      // Handle e.g.: (.. OR ..) OR (.. AND ...) 
      | (~GRconj(lx1,lx2), RX) => let
        val RXc = GREXP_copy(RX)
        in GRconj(toCNF(GRdisj(lx1,RX),emap),
          toCNF(GRdisj(lx2,RXc),emap)) 
        end
      | (LX ,~GRconj(rx1,rx2)) => let
        val LXc = GREXP_copy(LX)
        in GRconj(toCNF(GRdisj(LX,rx1),emap),
          toCNF(GRdisj(LXc,rx2),emap)) 
        end
      | (~GRconj(gc), RX) => let
        val retGR = toCNF(conjunctivize(RX, gc,emap),emap)
        val _ = genes_free(gc)
        in retGR end
      | (LX, ~GRconj(gc)) => let
        val retGR = toCNF (conjunctivize(LX, gc,emap),emap)
        val _ = genes_free(gc)
        in retGR end
      // All other disjunctive cases
      | (_,_) => GRdisj(toCNF(ex1,emap),toCNF(ex2,emap))
      ):GREXP
    | EX => EX
    ):GREXP
  end
\end{verbatim}

\subsection{Parsing functions and other GPR tools}
\label{ssec:parsing}

GPR rules were traditionally used to evaluate whether a gene deletion
would result in an enzyme complex being knocked out and the resulting
lower and upper bounds on the associated reaction being set to
zero. As we have shown that GPR rules can be used much more
quantitatively, it may be worth standardizing the form GPR rules
take. Towards that end, the software we have implemented for
Algorithm~\ref{alg:ReductionToCNF} has been developed using a modular
programming approach and can be used from any language that supports
a C API, making it easy for the user to experiment
with. A simple example is that the parser is a separate function from
the function implementing Algorithm~\ref{alg:ReductionToCNF}, allowing
the user to only work at the parsing level if estimating expression
levels is not the goal, or to substitute a different algorithm in in
place of Algorithm~\ref{alg:ReductionToCNF}. Additionally, warning
messages can be readily introduced in the parser for different
scenarios that may not be desirable. Currently, line and column
numbers of parsing errors are emitted, helping to quickly pinpoint any
errors that occur in GPR rule annotation. Some examples of using
the parser are given (\suppOrApp Section~\ref{ssec:parsing}).

\hl{Need to include examples on how to construct a GREXP, how to call
parser on de novo or IO-based GPR rule.}


\section{FALCON internals and experimental features}
\label{sec:internals}

\subsection{EXPCON: expression constraints for $V_{max}$}
An experimental feature that has been implemented is to constrain
enzymatic fluxes to be strictly less than or equal to the
automatically scaled enzyme abundance. For example, in the
notation of Algorithm~\ref{alg:FALCON}, for each reversible $v_j$
associated to enzyme complex $i$, we would have the additional
constraints:

\[ v_{j,f} \leq n e_i \]
\[ v_{j,b} \leq n e_i \]

This feature may be employed by setting the \texttt{EXPCON} parameter
to \texttt{true} when calling \texttt{falcon}. We implemented this
feature because we have seen and heard of others seeing promising
results when na\"ively using scaled expression to act as a surrogate for
$V_{max}$ in standard FBA (for an abstraction and example, see
\citealt{Colijn2009}). In the present study, we observed similar
results when using this parameter, suggesting that---at least for the
yeast models---this parameter may provide additional benefit when
appended to the existing FALCON machinery. We speculate that larger
and less constrained models may benefit from employing this parameter.

\subsection{Regularization}
Although sometimes called by different names, regularization is an
optimization technique that alters a problem so that there is some
penalty for variables taking on large values, where large may be
defined in different ways. In our case, since we are using a linear
programming problem, it is most convenient to use the $L_1$-norm to
assess how large a penalty should be given. We employ this as an
optional parameter to \texttt{falcon}, \texttt{rc} that is applied to
all flux values.  Regularization has been found to be a biologically
important objective in microbes \citep{Schuetz2012}, but we didn't
find any significant differences when using it in the present study.

\subsection{Growth rates}
Sometimes a growth rate may be known or we may wish to simulate a
growth rate. If we have a high-confidence biomass pseudo-reaction in
our model, the optional \texttt{minFit} parameter to \texttt{falcon}
can be used to set the minimum growth rate. Since biomass is a
complicated sink in the model, it is apparently unlikely that the
FALCON objective will direct flux into biomass directly, so other
measures of growth rate may also be warranted, such as accounting
for individual sink reactions. This part of \texttt{falcon} could
easily be modified to support more reactions beyond biomass.

\subsection{Debugging and developing falcon}
The \texttt{FDEBUG} parameter to \texttt{falcon} can be used to
display additional information while the FALCON algorithm is
running. As this can potentially be quite verbose, \texttt{FDEBUG} is
normally set to \texttt{false}. \texttt{FDEBUG} is also a parameter to
\texttt{computeMinDisj}, the MATLAB wrapper function for the
\texttt{minDisj} executable, and is used similarly there.


\subsection{Including reversible reactions in the FALCON objective}
Since FALCON has been designed for use with irreversible models, there
is no mathematical problem with including both the forward and
backward reactions of a reversible reaction in the FALCON objective,
but this may give undesired results in some cases, such as cycles
between the forward and backward reactions (the cycles are not a
problem directly because they can be removed: see the functions
\texttt{setFBRxnDirection} in \texttt{falcon.m}.  These cycles can
then give rise to objective values that may be quite high simply due
to having many cycles.

%\subsection{Linear Fractional Program (Charnes-Cooper) Transformation}
%Since our denominator is simply the variable $n$, it is always
%required to be equal to 1 in the transformed problem. However,
%this is scaled by $z$, which may vary. To be sure that $z$ never
%goes to zero, we set a lower about on $z$ with the \texttt{ZMIN}
%variable. We know $z = 0$ is undesirable in our case since
%we have $z = \frac{1}{n}$ and $v_original = n v_{LFP} = \frac{v_{LFP}}{z}$.
%!! Actually, we should not need this, since dividing by small z would make for 
% a very large original objective value. This may only be needed in cases
% where we aren't using a true LFP.

\subsection{LP solver settings}
\label{ssec:lpsettings}
We have exclusively used the Gurobi solver \citep{gurobi} for this
work, which is a highly competitive solver that employs by default a
parallel strategy to solving problems: a different algorithm is run
simultaneously, and as soon as one algorithm finished the others
terminate. Of course, if there is a clear choice of algorithm for a
particular problem class, this should be used in production settings
to avoid wasted CPU time and memory. In order to address this, we
benchmarked the three non-parallel solver methods in Gurobi
 (since parallel solvers simply use multiple methods simultaneously).
The exception to this rule is the Barrier method, which can use
multiple threads, but in practice for our models appears to use
no more than about 6 full CPU cores simultaneously for our models.
Our results for Yeast 5 and Yeast 7 with minimal directionality constraints
\citep{Heavner2012,Lee2012,Aung2013} and Human Recon 2 \citep{Thiele2013}
are shown in Table~\ref{tab:methodTime}).

\begin{table}
\begin{center}
\begin{tabular}{rrrr}
\emph{Model}                 & \emph{Primal-Simplex} & \emph{Dual-Simplex} & \emph{Barrier} \\
Yeast 5.21 (2,061 reactions) & $ 7.841 \pm 1.697    $ & $ 7.611 \pm 1.267    $ & $ 10.859 \pm 2.788   $\\ 
Yeast 7.0 (3,498 reactions)  & $ 51.863 \pm 22.731  $ & $ 65.317 \pm 12.771  $ & $ 242.137 \pm 57.129 $\\
Human 2.03 (7,440 reactions) & $ 159.077 \pm 24.903 $ & $ 152.297 \pm 39.783 $ & $ 366.166 \pm 92.321 $\\
\end{tabular}
\end{center}
\caption{Running times (in seconds, $\pm$ standard deviation) for
  FALCON using various algorithms implemented in the Gurobi package.
  For yeast models, 1,000 replicates were performed, and for the human
  model, 100 replicates were performed.}
\label{tab:methodTime}
\end{table}

We found that in Yeast7 with the primal-simplex solver, there is a
chance the solver will fail to find a feasible solution.
We verified that this is a numeric issue
in Gurobi and can be fixed by setting the Gurobi parameter
\texttt{MarkowitzTol} to a larger value (which decreases
time-efficiency but limits the numerical error in the
simplex algorithm). In practice, failure for the algorithm to converge
at an advanced iteration is rare and is not always a major problem (since the previous
flux estimate by the advanced iteration should already be quite good), but it
is certainly undesirable; a warning message will be printed by
\texttt{falcon} if this occurs, at which point parameter settings can
be investigated. In the future, we plan to improve \texttt{falcon} so
that parameters will be adjusted as needed during progression of the
algorithm after finding a good test suite of models and data. For now,
we use the dual-simplex solver, for which we have always had good
results.

Because the number of iterations depends non-trivially on the model
and the expression data, it may be more helpful to look at the 
average time per iteration in the above examples (Table~\ref{tab:methodTimeIter}).

\begin{table}
\begin{center}
\begin{tabular}{rrrr}
\emph{Model}                 & \emph{Primal-Simplex} & \emph{Dual-Simplex} & \emph{Barrier} \\
Yeast 5.21 (2,061 reactions) & $ 0.721 \pm 0.023 $ & $ 0.652 \pm 0.040 $ & $ 1.100 \pm 0.112  $\\ 
Yeast 7.0 (3,498 reactions)  & $ 2.725 \pm 0.298 $ & $ 2.469 \pm 0.289 $ & $ 11.309 \pm 1.589 $\\
Human 2.03 (7,440 reactions) & $ 6.422 \pm 0.484 $ & $ 5.233 \pm 0.661 $ & $ 15.782 \pm 3.209 $\\ 
\end{tabular}
\end{center}
\caption{Running time per FALCON iteration (in seconds, $\pm$ standard
  deviation) using various algorithms implemented in the Gurobi
  package.  For yeast models, 1,000 replicates were performed, and for
  the human model, 100 replicates were performed.}
\label{tab:methodTimeIter}
\end{table}

Given the above rare trouble with primal simplex solver the universal
best performance enjoyed by the dual-simplex method (Table~\ref{tab:methodTime},
Table~\ref{tab:methodTimeIter}), we would advise the dual-simplex algorithms, all else
being equal. The dual-simplex method is also recommended for
memory-efficiency by Gurobi documentation, but we did not observe any
differences in memory for different solver methods.

\hl{keep this paragraph}
It is not possible to pass the Gurobi \texttt{method} parameter to
\texttt{gurobi} in the COBRA Toolbox by default, but to keep our code
as close as possible to the COBRA Toolbox API, we have copied
\texttt{solveCobraLP.m} to the file \texttt{solveFalconLP.m} and
modified it to use the optimal method parameters. Other parameters
could also be passed easily. \texttt{solveFalconLP} is only called if
\texttt{gurobi} is detected as a MATLAB executable (MEX) file;
otherwise the default solver specified for the COBRA Toolbox will be
used. At the time of this release, \texttt{solveFalconLP} should be
completely compatible with all other possible calls made to
\texttt{solveCobraLP}, so in principle the latter could be replaced,
but it is not recommended in case the COBRA Toolbox API changes. If a
user wishes to use some solver other than Gurobi, it should be easy to
alter \texttt{solveFalconLP.m} and \texttt{falcon.m} to support custom
parameters. 

All timing analyses were performed on a system with four 8-core AMD
Opteron\texttrademark\ 6136 processors operating at 2.4 GHz. Figure
FluxBars, and Tables FalcPerf, \ref{tab:methodTime} and
\ref{tab:methodTimeIter} used a single unperturbed expression file per
species (\textit{S. cerevisiae}, \textit{E. coli}, and
\textit{H. sapiens}; see \texttt{timingAnalysis.m} for
details). Values were averaged across 32
replicates. Tables~\ref{tab:methodTime} and \ref{tab:methodTimeIter}
used multivariate log-normal noise multiplied by the original
expression vector (refer to Section~\ref{sec:noise} for details) to
introduce more variance in the calculations; the human models were
tested with 100 replicates and the yeast models with 500 replicates.
