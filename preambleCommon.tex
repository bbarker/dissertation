% It seems that we need to use
% \usepackage[dvipfmx]{graphicx}, which is
% incompatible with some other package being
% used here.
%
%\def\pgfsysdriver{pgfsys-dvipdfmx.def}

\usepackage{caption}

%\usepackage{hangcaption}
%\usepackage{algorithmic}
%\usepackage{algorithmicx}
\usepackage{algorithm}
\usepackage{algpseudocode}
\usepackage{float}
\usepackage{ifthen}
\usepackage{soul} %For highlights
\usepackage{tikz}
\usepackage[export]{adjustbox}
\usepackage{placeins}

%To help keep figures and tables in order:
\usepackage{fixltx2e}
%To fix quotes in verbatim, etc:
\usepackage{upquote,textcomp}

%Some possible packages to include
\usepackage{epsfig}
\usepackage{graphicx,pstricks}
\usepackage{palatino}
%\usepackage{subfigure}
\usepackage{subcaption}
\usepackage{txfonts}


\graphicspath{{./figures/}}

\usetikzlibrary{shapes, arrows}

% Some black magic:
\makeatletter
\newlength{\parskipsave}
\newcommand\floatc@plainthin[2]{
\setbox\@tempboxa\hbox{{\@fs@cfont #1:} #2}
   \hbox to\hsize{\hfil\box\@tempboxa\hfil}\fi}
\newcommand\fs@plainthin{
  \def\@fs@cfont{\rmfamily}\let\@fs@capt\floatc@plainthin
  \def\@fs@pre{}
  \def\@fs@post{\vspace{-2.5em}}
  \def\@fs@mid{\vspace\abovecaptionskip\relax}
  \let\@fs@iftopcapt\iffalse}
\makeatother
\floatstyle{plainthin}
\newfloat{AlgFloat}{h}{lop}


%\renewcommand{\algorithmicrequire}{\textbf{Input:}}
%\renewcommand{\algorithmicensure}{\textbf{Output:}}
%\newcommand{\INDSTATE}[1][1]{\STATE\hspace{#1\algorithmicindent}}
\algnewcommand\algorithmicinput{\textbf{INPUT:}}
\algnewcommand\INPUT{\item[\algorithmicinput]}
\algnewcommand\algorithmicoutput{\textbf{OUTPUT:}}
\algnewcommand\OUTPUT{\item[\algorithmicoutput]}

\newtheorem{Theorem}{Theorem}
\newtheoremstyle{break}  % follow `plain` defaults but change HEADSPACE.
  {\topsep}   % ABOVESPACE
  {\topsep}   % BELOWSPACE
  {\itshape}  % BODYFONT
  {-1em}      % INDENT (empty value is the same as 0pt)
  {\bfseries} % HEADFONT
  {}         % HEADPUNCT
  {0pt}  % HEADSPACE. `plain` default: {5pt plus 1pt minus 1pt}
  {}          % CUSTOM-HEAD-SPEC
\theoremstyle{break}
\newtheorem{Algorithm}{Algorithm}

\DeclareMathOperator*{\argmin}{arg\,min}
\newcommand{\E}[1]{\operatorname{E}\left(#1\right)}
\newcommand{\sgn}[1]{\operatorname{sgn}\left(#1\right)}