% Template for PLoS
% Version 2.0 July 2014
%
% To compile to pdf, run:
% latex plos.template
% bibtex plos.template
% latex plos.template
% latex plos.template
% dvipdf plos.template
%
% % % % % % % % % % % % % % % % % % % % % %
%
% -- IMPORTANT NOTE
%
% Be advised that this is merely a template 
% designed to facilitate accurate translation of manuscript content 
% into our production files. 
%
% This template contains extensive comments intended 
% to minimize problems and delays during our production 
% process. Please follow the template 
% whenever possible.
%
% % % % % % % % % % % % % % % % % % % % % % % 
%
% Once your paper is accepted for publication and enters production, 
% PLEASE REMOVE ALL TRACKED CHANGES in this file and leave only
% the final text of your manuscript.
%
% DO NOT ADD EXTRA PACKAGES TO THIS TEMPLATE unless absolutely necessary.
% Packages included in this template are intentionally
% limited and basic in order to reduce the possibility
% of issues during our production process.
%
% % % % % % % % % % % % % % % % % % % % % % %
%
% -- FIGURES AND TABLES
%
% DO NOT INCLUDE GRAPHICS IN YOUR MANUSCRIPT
% - Figures should be uploaded separately from your manuscript file. 
% - Figures generated using LaTeX should be extracted and removed from the PDF before submission. 
% - Figures containing multiple panels/subfigures must be combined into one image file before submission.
% See http://www.plosone.org/static/figureGuidelines for PLOS figure guidelines.
%
% Tables should be cell-based and may not contain:
% - tabs/spacing/line breaks within cells to alter layout
% - vertically-merged cells (no tabular environments within tabular environments, do not use \multirow)
% - colors, shading, or graphic objects
% See http://www.plosone.org/static/figureGuidelines#tables for table guidelines.
%
% For sideways tables, use the {rotating} package and use \begin{sidewaystable} instead of \begin{table} in the appropriate section. PLOS guidelines do not accomodate sideways figures.
%
% % % % % % % % % % % % % % % % % % % % % % % %
%
% -- EQUATIONS, MATH SYMBOLS, SUBSCRIPTS, AND SUPERSCRIPTS
%
% IMPORTANT
% Below are a few tips to help format your equations and other special characters according to our specifications. For more tips to help reduce the possibility of formatting errors during conversion, please see our LaTeX guidelines at http://www.plosone.org/static/latexGuidelines
%
% Please be sure to include all portions of an equation in the math environment, and for any superscripts or subscripts also include the base number/text. For example, use $mathrm{mm}^2$ instead of mm$^2$ (do not use \textsuperscript command).
%
% DO NOT USE the \rm command to render mathmode characters in roman font, instead use $\mathrm{}$
% For bolding characters in mathmode, please use $\mathbf{}$ 
%
% Please add line breaks to long equations when possible in order to fit our 2-column layout. 
%
% For inline equations, please do not include punctuation within the math environment unless this is part of the equation.
%
% For spaces within the math environment please use the \; or \: commands, even within \text{} (do not use smaller spacing as this does not convert well).
%
%
% % % % % % % % % % % % % % % % % % % % % % % %



\documentclass[10pt]{article}

% amsmath package, useful for mathematical formulas
\usepackage{amsmath}
% amssymb package, useful for mathematical symbols
\usepackage{amssymb}

% cite package, to clean up citations in the main text. Do not remove.
\usepackage{cite}

\usepackage{hyperref}

% line numbers
\usepackage{lineno}

% ligatures disabled
\usepackage{microtype}
\DisableLigatures[f]{encoding = *, family = * }

% rotating package for sideways tables
%\usepackage{rotating}

% If you wish to include algorithms, please use one of the packages below. Also, please see the algorithm section of our LaTeX guidelines (http://www.plosone.org/static/latexGuidelines) for important information about required formatting.
%\usepackage{algorithmic}
%\usepackage{algorithmicx}

% Use doublespacing - comment out for single spacing
%\usepackage{setspace} 
%\doublespacing


% Text layout
\topmargin 0.0cm
\oddsidemargin 0.5cm
\evensidemargin 0.5cm
\textwidth 16cm 
\textheight 21cm

% Bold the 'Figure #' in the caption and separate it with a period
% Captions will be left justified
\usepackage[labelfont=bf,labelsep=period,justification=raggedright]{caption}

% Use the PLoS provided BiBTeX style
\bibliographystyle{plos2009}

% Remove brackets from numbering in List of References
\makeatletter
\renewcommand{\@biblabel}[1]{\quad#1.}
\makeatother


% Leave date blank
\date{}

\pagestyle{myheadings}

%% Include all macros below. Please limit the use of macros.

\def\D{\mathrm{d}}

\newcommand{\Fig}{Fig.}
\newcommand{\Figs}{Fig.} % It is the same for PLOS

%% END MACROS SECTION


\begin{document}


% Title must be 150 characters or less
\begin{flushleft}
{\Large
\textbf{Title}
}
% Insert Author names, affiliations and corresponding author email.
\\
Author1$^{1}$, 
Author2$^{2}$, 
Author3$^{3,\ast}$
\\
\bf{1} Author1 Dept/Program/Center, Institution Name, City, State, Country
\\
\bf{2} Author2 Dept/Program/Center, Institution Name, City, State, Country
\\
\bf{3} Author3 Dept/Program/Center, Institution Name, City, State, Country
\\
$\ast$ E-mail: Corresponding author@institute.edu
\end{flushleft}

% Please keep the abstract between 250 and 300 words
\section*{Abstract}

% Please keep the Author Summary between 150 and 200 words
% Use first person. PLOS ONE authors please skip this step. 
% Author Summary not valid for PLOS ONE submissions.   
\section*{Author Summary}



\section*{Introduction}




% You may title this section "Methods" or "Models". 
% "Models" is not a valid title for PLoS ONE authors. However, PLoS ONE
% authors may use "Analysis" 
\section*{Materials and Methods}

% Results and Discussion can be combined.
\section*{Results}

% We only support three levels of headings, please do not create a heading level below \subsubsection.
\subsection*{Subsection 1}

\subsubsection*{SubSubsection 1.1}

\subsection*{Subsection 2}

\section*{Discussion}



% Do NOT remove this, even if you are not including acknowledgments.

\section*{Acknowledgments}


\section*{References}

% Either type in your references using
% \begin{thebibliography}{}
% \bibitem{}
% Text
% \end{thebibliography}
%
% OR
%
% Compile your BiBTeX database using our plos2009.bst
% style file and paste the contents of your .bbl file
% here.
% 

\section*{Figure Legends}
% This section is for figure legends only, do not include
% graphics in your manuscript file.
%
%\begin{figure}
%\caption{
%{\bf Bold the first sentence.}  Rest of figure caption.  
%}
%\label{Figure_label}
%\end{figure}

\begin{figure}
\caption{
{\bf More positive differential epistases under environmental
perturbations.} (\textbf{A}) Heat maps describe the global dynamics of
differential epistasis from abundant-glucose medium to ethanol (left
panel) and glycerol (right panel) conditions. Only gene pairs with
$\left|\D\epsilon\right| \geq 0.01$ in either condition are included
in the figure. Different colors represent differential epistasis
values as indicated by the color bar at the bottom. The differential
epistasis values are assigned to be 0.1 (or -0.1) in the heat-maps
when it is greater than 0.1 (or less than -0.1). It is noteworthy to
point out that the epistasis patterns are indeed very different
between the two conditions (\Fig~\ref{fig:eef2}A). (\textbf{B})
Percentage of positive and negative differential epistases under
ethanol and glycerol conditions. (\textbf{C}) Ratio of positive to
negative differential epistases
in each simulated condition. The result from a high-throughput
experiment is also shown. The letters A-P represent acetaldehyde,
acetate, adenosine 3',5'-bisphosphate, adenosyl methionine, adenosine,
alanine, allantoin, arginine, ethanol, glutamate, glutamine, glycerol,
low glucose, phosphate, trehalose, and xanthosine, respectively.
}
\label{fig:eef1}
\end{figure}

\begin{figure}
\caption{
{\bf Epistasis dynamics between environmental perturbations.} (\textbf{A}) 
Number of gene pairs with various epistatic relationships between ethanol and
glycerol growth conditions. (\textbf{B}) The distribution for the percentages
of gene pairs with similar epistasis relation between any 2 of 16
conditions. The frequency is derived from the 120 pairs of
environments simulated in this study.
}
\label{fig:eef2}
\end{figure}


\begin{figure}
\caption{
{\bf The global distribution of epistatic relations under simulated
conditions.} (\textbf{A}) Distribution for the number of conditions in which
each epistatic interaction exists. Note that about 28\% of
epistatic relations are extremely stable (the very right bar) and
about 24\% are extremely dynamic (the very left bar). (\textbf{B}) Fraction
of three types of epistatic relations in each of the 16 environmental
perturbations, as indicated by the color bar to the right. The numbers
in the brackets represent the number of conditions in which each
epistatic interaction exists, as indicated in (\textbf{A}). The letters A-P
represent the simulated conditions as indicated in \Fig~\ref{fig:eef1}.
}
\label{fig:eef3}
\end{figure}

\begin{figure}
\caption{
{\bf Network properties for the extremely stable and extremely dynamic
epistatic interactions.} (\textbf{A}) Degree distribution for genes in two
epistatic interaction networks. The networks have nodes that
correspond to genes and edges that correspond to epistatic
interactions. (\textbf{B}) Three network parameters (the definition of which
are shown in Methods) for two epistatic interaction networks.
}
\label{fig:eef4}
\end{figure}

\begin{figure}
\caption{
{\bf Co-evolution between genes with epistasis.} (\textbf{A}) Average evolutionary
rate differences between gene pairs with FBA-predicted epistasis
(green), extremely dynamic epistasis (blue) and extremely stable
epistasis (red) are highlighted by three arrows, respectively. The
random simulations with the same number of gene pairs as each of the
three groups were repeated 10,000 times and the frequency
distributions are shown (marked by the same colors as the
corresponding arrows, respectively). (\textbf{B}) The evolutionary rates for
genes that are involved in extremely stable and extremely dynamic
epistasis, respectively. The error bars represent standard errors.
}
\label{fig:eef5}
\end{figure}


\section*{Tables}
% 
% See introductory notes if you wish to include sideways tables.
%
% NOTE: Please look over our table guidelines at http://www.plosone.org/static/figureGuidelines#tables to make sure that your tables meet our requirements. Certain types of spacing, cell merging, and other formatting tricks may have unintended results and will be returned for revision.
%
%\begin{table}[!ht]
%\caption{
%\bf{Table title}}
%\begin{tabular}{|c|c|c|}
%table information
%\end{tabular}
%\begin{flushleft}Table caption
%\end{flushleft}
%\label{tab:label}
% \end{table}

\section*{Supporting Information Legends}
%
% Please enter your Supporting Information captions below in the following format:
%\item{\bf Figure SX. Enter mandatory title here.} Enter optional descriptive information here.
% 
%\begin{description}
%\item {\bf}
%\item {\bf}
%\end{description}

\begin{description}
\item {\bf Figure S1. More positive differential epistases under environmental perturbations
for different thresholds of differential epistasis}
($\left|\D\epsilon\right| \geq 0.001$, \textbf{A}) and
($\left|\D\epsilon\right| \geq 0.05$, \textbf{B}). Ratio of positive
to negative differential epistases in each simulated condition are
shown. The letters A-P represent acetaldehyde, acetate, adenosine
3',5'-bisphosphate, adenosyl methionine, adenosine, alanine,
allantoin, arginine, ethanol, glutamate, glutamine, glycerol, low
glucose, phosphate, trehalose, and xanthosine, respectively.

\item {\bf Figure S2. Analogous to \Fig~\ref{fig:eef1}B-C, but using a
maximum growth rate for each condition, where the maximum is
constrained to be no higher than the high-glucose growth rate.}
(\textbf{A}) Percentage of positive and negative differential
epistases under ethanol and glycerol conditions. (\textbf{B}) Ratio of
positive to negative differential epistases in each simulated
condition. The result from a high-throughput experiment is also
shown. The letters A-P represent acetaldehyde, acetate, adenosine
3',5'-bisphosphate, adenosyl methionine, adenosine, alanine,
allantoin, arginine, ethanol, glutamate, glutamine, glycerol, low
glucose, phosphate, trehalose, and xanthosine, respectively. Note that
in (B), low glucose has the same growth rate as high-glucose, but has
different epistatic interactions since we still use the high-oxygen
uptake level associated with the low glucose condition.

\item {\bf Figure S3. Epistasis dynamics between environmental
perturbations under different epistasis definition.} (\textbf{A})
Number of gene pairs with various epistatic relationships between
ethanol and glycerol growth conditions under a lower
($\left|\epsilon\right| \geq 0.001$) and a higher
($\left|\epsilon\right| \geq 0.05$) epistasis threshold. (\textbf{B})
The distribution for the percentages of gene pairs with similar
epistasis relations between any 2 of 16 conditions under a lower
($\left|\epsilon\right| \geq 0.001$) and a higher
($\left|\epsilon\right| \geq 0.05$) epistasis threshold.

\item {\bf Figure S4. Analogous to \Fig~\ref{fig:eef3}A, but using a
maximum growth rate for each condition, where the maximum is
constrained to be no higher than the high-glucose growth rate.}
Distribution for the number of conditions in which each epistatic
interaction exists. Note that about 26\% of epistatic relations are
extremely stable (the very right bar) and about 19\% are extremely
dynamic (the very left bar). 

\item {\bf Table S1. Wild-type growth rates used in the maximal growth
rate simulations used for \Figs S2 and S4.}

\item {\bf Table S2. Condition-specific epistases and sign-epistases
prevalence in the iMM904 yeast model.}

\item {\bf Table S3. GO term enrichment analysis results for
differential epistasis in transition to ethanol.}

\item {\bf Table S4. Properties of simulated systems that correlate
with the ratio of positive to negative differential epistases.}

\item {\bf Table S5. List of epistatic interactions for the extremely
stable, dynamic, and intermediate epistasis networks.}

\item {\bf Table S6. Table of network parameters for stable, dynamic,
and intermediate epistasis.}


\end{description}

\end{document}

