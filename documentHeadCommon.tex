% Define block styls
\tikzstyle{line} = [draw, very thick, color=black!50, -latex']
\tikzstyle{cloud} = [draw, ellipse,fill=red!20, node distance=2em]
\tikzstyle{decision} = [diamond, draw, fill=blue!20,
    text badly centered, node distance=6em, inner sep=0pt]
\tikzstyle{block} = [rectangle, draw, fill=blue!20,
    text centered, node distance=4em]
\tikzstyle{iogram} = [trapezium, draw, fill=pink!20,
    trapezium left angle=70, trapezium right angle=-70,
    text centered, align=center, node distance=5em]

\newcommand{\suppOrApp}{
  \ifthenelse{\boolean{thesisStyle}}
    {Appendix}
    {Supplementary Information}
}

\newcommand{\falconAbstractMotivation}{
A major theme in constraint-based modeling is unifying small-scale
experimental data, such as biochemical information about the reactions
that can occur in a system or the composition and localization of enzyme
complexes, with high-throughput data including expression data,
metabolomics, or DNA sequencing. The desired result is to improve
understanding about the metabolic pathways that are being used in a
specific organism, type of cell, or environmental condition. The
approach typically employed when only gene (or protein) intensities
are available is the creation of tissue-specific models, which reduces
the available reactions in an organism model, and does not provide an
objective function for the estimation of fluxes, which is an important
limitation in many modeling applications.
}

\newcommand{\falconAbstractResults}{
We develop a method, flux assignment with LAD (least absolute
deviation) convex objectives and normalization (FALCON), based on a
technique originally applied in Yeast that employs metabolic network
reconstructions along with expression
data that estimates fluxes without needing to reduce the reaction set
of the model.  Our extensions to the original technique include the
capability to work with large models, an improved analysis of enzyme
complex formation logic, the ability to handle very large complex
rules that may incorporate multiple isoforms, significantly improved
runtime, and depending on the model constraints, either maintained or
significantly improved correlation with experimentally measured
fluxes.
}
