\documentclass[compress,red]{beamer}
\mode<presentation>

\usetheme{Warsaw}
% other themes: AnnArbor, Antibes, Bergen, Berkeley, Berlin, Boadilla, boxes, CambridgeUS, Copenhagen, Darmstadt, default, Dresden, Frankfurt, Goettingen,
% Hannover, Ilmenau, JuanLesPins, Luebeck, Madrid, Maloe, Marburg, Montpellier, PaloAlto, Pittsburg, Rochester, Singapore, Szeged, classic

%\usecolortheme{lily}
% color themes: albatross, beaver, beetle, crane, default, dolphin, dov, fly, lily, orchid, rose, seagull, seahorse, sidebartab, structure, whale, wolverine

%\usefonttheme{serif}
% font themes: default, professionalfonts, serif, structurebold, structureitalicserif, structuresmallcapsserif

\hypersetup{pdfpagemode=FullScreen} % makes your presentation go automatically to full screen

% define your own colors:
\definecolor{CURed}{rgb}{0.702,0.106,0.106}
\definecolor{CUGrey}{rgb}{0.302,0.31,0.3255}
%
\definecolor{White}{rgb}{1,1,1}
\definecolor{Red}{rgb}{1,0,0}
\definecolor{Blue}{rgb}{0,0,1}
\definecolor{Green}{rgb}{0,1,0}
\definecolor{magenta}{rgb}{1,0,.6}
\definecolor{lightblue}{rgb}{0,.5,1}
\definecolor{lightpurple}{rgb}{.6,.4,1}
\definecolor{gold}{rgb}{.6,.5,0}
\definecolor{orange}{rgb}{1,0.4,0}
\definecolor{hotpink}{rgb}{1,0,0.5}
\definecolor{newcolor2}{rgb}{.5,.3,.5}
\definecolor{newcolor}{rgb}{0,.3,1}
\definecolor{newcolor3}{rgb}{1,0,.35}
\definecolor{darkgreen1}{rgb}{0, .35, 0}
\definecolor{darkgreen}{rgb}{0, .6, 0}
\definecolor{darkred}{rgb}{.75,0,0}

\xdefinecolor{olive}{cmyk}{0.64,0,0.95,0.4}
\xdefinecolor{purpleish}{cmyk}{0.75,0.75,0,0}

% can also choose different themes for the "inside" and "outside"

% \usepackage{beamerinnertheme_______}
% inner themes include circles, default, inmargin, rectangles, rounded

% \usepackage{beamerouterthemesmoothbars}
% outer themes include default, infolines, miniframes, shadow, sidebar, smoothbars, smoothtree, split, tree

\useoutertheme[subsection=false]{smoothbars}

% to have the same footer on all slides
%\setbeamertemplate{footline}[text line]{STUFF HERE!}
\setbeamertemplate{footline}[text line]{} % makes the footer EMPTY

% Change some colors:
\setbeamercolor{structure}{fg=CURed}
\setbeamercolor*{palette quaternary}{fg=white,bg=CUGrey}

% include packages
\usepackage{subfigure}
\usepackage{multicol}
\usepackage{amsmath}
\usepackage{epsfig}
\usepackage{graphicx}
\usepackage[all,knot]{xy}
\xyoption{arc}
\usepackage{url}
\usepackage{multimedia}
\usepackage{hyperref}
     
%%%%%%%%%%%%%%%%%%%%%%%%%%%%%%%%%%%%%%%%%%%%%%%%%%%%%%%%%%%%%%%%%%%%%%%%%%%%%%%%%%%%%%%%%%
%%%%%%%%%%%%%%%%%%%%%%%%%%%%%% Title Page Info %%%%%%%%%%%%%%%%%%%%%%%%%%%%%%%%%%%%%%%%%%%
%%%%%%%%%%%%%%%%%%%%%%%%%%%%%%%%%%%%%%%%%%%%%%%%%%%%%%%%%%%%%%%%%%%%%%%%%%%%%%%%%%%%%%%%%%

\title{Estimating Metabolic Fluxes from Gene Expression}
\subtitle{A Principled Approach}
\author{Brandon Barker}
\institute{Department of Biological Statistics and Computational Biology\\ 
Cornell University \\ \vspace{.25cm}CAC Job Search}
\date{\today}

%%%%%%%%%%%%%%%%%%%%%%%%%%%%%%%%%%%%%%%%%%%%%%%%%%%%%%%%%%%%%%%%%%%%%%%%%%%%%%%%%%%%%%%%%%
%%%%%%%%%%%%%%%%%%%%%%%%%%%%%% Begin Your Document %%%%%%%%%%%%%%%%%%%%%%%%%%%%%%%%%%%%%%%
%%%%%%%%%%%%%%%%%%%%%%%%%%%%%%%%%%%%%%%%%%%%%%%%%%%%%%%%%%%%%%%%%%%%%%%%%%%%%%%%%%%%%%%%%%

\begin{document}

%%%%%%%%%%%%%%%%%%%%%%%%%%%%%%%%%%%%%%%%%%%%%%%%%%%%%%%%%%%%%%%%%%%%%%%%%%%%%%%%%%%%%%%%%%

\frame{
	\titlepage 
}

%%%%%%%%%%%%%%%%%%%%%%%%%%%%%%%%%%%%%%%%%%%%%%%%%%%%%%%%%%%%%%%%%%%%%%%%%%%%%%%%%%%%%%%%%%
% this puts the outline before EACH section automatically & will
% highlight the section you're about to talk about
%\section[Outline]{}
%\frame{\tableofcontents}

%%%%%%%%%%%%%%%%%%%%%%%%%%%%%%%%%%%%%%%%%%%%%%%%%%%%%%%%%%%%%%%%%%%%%%%%%%%%%%%%%%%%%%%%%%

\section{Introduction}
\subsection{Beamer Intro}

%%%%%%%%%%%%%%%%%%%%%%%%%%%%%%%%%%%%%%%%%%%%%%%%%%%%%%%%%%%%%%%%%%%%%%%%%%%%%%%%%%%%%%%%%%

\frame{\frametitle{}
\begin{center}
\begin{block}<+->{Disclaimer \#1}
	\vspace{.1cm}
	\begin{center} \large
	I am \textcolor{darkgreen}{NOT} an expert in \LaTeX\\ \vspace{.1cm}
	I am \textcolor{darkgreen}{NOT} an expert in Beamer\\
	\end{center}
\end{block}
\vspace{1cm}
\begin{block}{Disclaimer \#2}
	\vspace{.1cm}
	This talk is designed to \textcolor{darkgreen}{introduce} you to presentations in \LaTeX\\ \vspace{.1cm}
	\begin{center}
	$\dots$ and showcase cool features of Beamer\\
	\end{center}
\end{block}
\end{center}
}

%%%%%%%%%%%%%%%%%%%%%%%%%%%%%%%%%%%%%%%%%%%%%%%%%%%%%%%%%%%%%%%%%%%%%%%%%%%%%%%%%%%%%%%%%%

\frame{\frametitle{Why Use \LaTeX \hspace{.05cm} for Presentations (and everything else)?}
\vspace{-.5cm}
\footnotesize
\center
\textsc{\textcolor{Blue}{Because Microsoft SUCKS!}}\\
\scriptsize
\flushleft
$\dots\dots$ especially for mathematics $\dots\dots$
\begin{columns}[t]
\column{.35\textwidth}
\begin{eqnarray*}
	\frac{\partial^2 u}{\partial t^2} &=& c^2 \nabla^2u\\
	\int\limits_{0}^{\infty} e^{-x}&=& 1
\end{eqnarray*}
\column{.65\textwidth}
\begin{eqnarray*}
	f(x) &=&a_o+\sum\limits_{n=1}^{\infty}\left[a_n \cos\left(\frac{n\pi x}{L}\right)+b_n\sin \left(\frac{n \pi x}{L}\right)\right]\\
        \Psi(x)&=&\left\{
        \begin{array}{ccr}
		1 & \mbox{if}& x<0\\
                \frac{x^2}{4} & \mbox{if}& x\ge 0
        \end{array} \right.
\end{eqnarray*}
\end{columns}
					% XY Diagram
\begin{columns}[t]
\column{.5\textwidth}
\xymatrix{
U\ar@/_/[ddr]_y \ar@/^/[drr]^x
 \ar@{.>}[dr]|-{(x,y)}    \\
 & X \times_Z Y \ar[d]^q \ar[r]_p
 & X \ar[d]_f \\
 & Y \ar[r]^g &Z
 }
\column{.5\textwidth}
\[ \xy					% XY Diagram
(6,9)*{}="1";
(-8.5,-1)*{}="2";
"1";"2" **\crv{~*=<.5pt>{.} (0,30)}?(.75)*\dir{>}+(-2,2)*{z};
(-6.5,8)*{}="1";
(-.5,-9)*{}="2";
"1";"2" **\crv{~*=<.5pt>{.} (-28.5,9.3)}?(.7)*\dir{>}+(-2,-2)*{u};
(-9.5,-3.35)*{}="1";
(8.5,-3)*{}="2";
"1";"2" **\crv{~*=<.5pt>{.} (-17.67,-24.19)}?(.7)*\dir{>}+(-1,-3)*{x};
(1,-10)*{}="1";
(6.5,7.13)*{}="2";
"1";"2" **\crv{~*=<.5pt>{.} (17.67,-24.19)}?(.7)*\dir{>}+(3,-1)*{y};
(11,-1)*{}="1";
(-4,8)*{}="2";
"1";"2" **\crv{~*=<.5pt>{.} (28.5,9.3)}?(.93)*\dir{>}+(1,2)*{w};
\endxy \]
\end{columns}
}

%%%%%%%%%%%%%%%%%%%%%%%%%%%%%%%%%%%%%%%%%%%%%%%%%%%%%%%%%%%%%%%%%%%%%%%%%%%%%%%%%%%%%%%%%%

\frame{\frametitle{For the Pure Mathematicians$\dots$}
\vspace{-.5cm}
\begin{center}
\LaTeX  \hspace{.05cm}can \textcolor{Blue}{DRAW} cool diagrams!
\end{center}
\begin{columns}[t]
\column{0.35\textwidth}
\xy 				%FIG.5. Describin a cobordism using Moorse Theory (BIG)
(0,10)*\ellipse(3,1){-};
(0,-12)*\ellipse(3,1)__,=:a(-180){-};
(0,-12)*\ellipse(3,1){.};
(0,7.5)*\ellipse(3,1){.};
(0,7.5)*\ellipse(3,1)__,=:a(-180){-};
(-3,0)*\ellipse(3,1){.};
(3,0)*\ellipse(3,1){.};
(-3,0)*\ellipse(3,1)__,=:a(-180){-};
(3,0)*\ellipse(3,1)__,=:a(-180){-};
(0,-6.5)*\ellipse(3,1){.};
(0,-6.5)*\ellipse(3,1)__,=:a(-180){-};
(-3,0)*{}="1";
(3,0)*{}="2";
(-9,0)*{}="xA2";
(9,0)*{}="xB2";
"1";"2" **\crv{(-3,5) & (3,5)};
(-3,15)*{}="xA";
(3,15)*{}="xB";
(-3,9)*{}="xA1";
(3,9)*{}="xB1";
"xA";"xA1" **\dir{-};
"xB";"xB1" **\dir{-};
"xB1";"xB2" **\crv{(8,7)};
"xA1";"xA2" **\crv{(-8,7)};
(-3,0)*{}="1";
(3,0)*{}="2";
(-9,0)*{}="A2";
(9,0)*{}="B2";
"1";"2" **\crv{(-3,-5) & (3,-5)};
(-3,-13)*{}="A";
(3,-13)*{}="B";
(-3,-9)*{}="A1";
(3,-9)*{}="B1";
"A";"A1" **\dir{-};
"B";"B1" **\dir{-};
"B1";"B2" **\crv{(8,-7)};
"A1";"A2" **\crv{(-8,-7)};

%TUBE TIP*********************

(-3,20)*{}="f1";
(-3,-19.5)*{}="f2";
(3,20)*{}="f3";
(3,-19.5)*{}="f4";
(-3,-24)*{}="xf2";
(3,-24)*{}="xf4";
"f2";"xf2" **\dir{-};
"f4";"xf4" **\dir{-};
"f1";"xA" **\dir{-};
"f3";"xB" **\dir{-};
"f2";"A" **\dir{-}; 	%ADJUST HERE IF PLANES ARE NOT TRANSPERANT
"f4";"B" **\dir{-};

%End tube tips***************

(0,-8)*{}="A"; 		%START OF SLICES
(18,0)*{}="B";
(0,8)*{}="G";
(-18,0)*{}="Y";
(-8,4.5)*{}="L";
(8,4.5)*{}="R";
"A";"B" **\dir{-};
"Y";"L" **\dir{-};
"Y";"A" **\dir{-};
"R";"B" **\dir{-};

%New SLICE

(0,7)*{}="A"; 		%START OF SLICES
(18,15)*{}="B";
(0,23)*{}="G";
(-18,15)*{}="Y";
"A";"B" **\dir{-};
"Y";"G" **\dir{-};
"Y";"A" **\dir{-};
"B";"G" **\dir{-};
(0,-21)*{}="A"; 	%START OF SLICES
(18,-13)*{}="B";
(0,-5)*{}="G";
(-18,-13)*{}="Y";
(6,-8)*{}="R";
(-6,-8)*{}="L";
"A";"B" **\dir{-};
"Y";"L" **\dir{-};
"Y";"A" **\dir{-};
"R";"B" **\dir{-};
\endxy			% End XY pic
\hfill
\column{0.3\textwidth}
\[
\xy				% XY Diagram
(-5,0)*{f};
(-5,12)*{}; (-5,0)*\xycircle(2.65,2.65){-}="1_x"; **\dir{-}
?(.5)*\dir{<}+(3,0)*{\scriptstyle x}; "1_x";(-5,-12)*{}; **\dir{-}
?(.4)*\dir{<}+(3,0)*{\scriptstyle y}; (5,12)*{};
(5,0)*\xycircle(2.65,2.65){-}="1_x"; **\dir{-}
?(.5)*\dir{<}+(3,0)*{\scriptstyle x'}; "1_x";(5,-12)*{}; **\dir{-}
?(.4)*\dir{<}+(3,0)*{\scriptstyle y'}; (5,0)*{g};
\endxy
\]
\hfill
\column{0.35\textwidth}
\xy 0;/r.20pc/:			% XY Diagram
(-6,9)*{}="1";
(8.5,-1)*{}="2";
"1";"2" **\crv{~*=<.5pt>{.}(0,30)};
(6.5,8)*{}="1";
(.5,-9)*{}="2";
"1";"2" **\crv{~*=<.5pt>{.}(28.5,9.3)}\POS?(.9)*+{\hole}="d"\POS?(.65)*{\hole}="e";
(9.5,-3.35)*{}="1";
(-8.5,-3)*{}="2";
"1";"2" **\crv{~*=<.5pt>{.}(17.67,-24.19)}\POS?(.9)*+{\hole}="b" \POS?(.7)*{\hole}="c";
(-1,-10)*{}="1";
(-6.5,7.13)*{}="2";
"1";"2" **\crv{~*=<.5pt>{.}(-17.67,-24.19)};
(-11,-1)*{}="1";
(4,8)*{}="2";
"1";"2" **\crv{~*=<.5pt>{.}(-28.5,9.3)}\POS?(.12)*+{\hole}="a";
(-20,-40)*{};"a" **\crv{~*=<2pt>{.} (-5,-35) & (-20,-5)};
"a";"b" **\crv{~*=<2pt>{.} (-10,5) & (0,2)};
"b";"c" **\crv{~*=<2pt>{.} (-10,-15) & (3,-20)};
"c";"d" **\crv{~*=<2pt>{.} (6,-10)};
"d";"e" **\crv{~*=<2pt>{.} (3,7) & (11,7)};
"e";(-10,-44) **\crv{~*=<2pt>{.} (30,-30) & (-11,-25)};
\endxy
\end{columns}
}

%%%%%%%%%%%%%%%%%%%%%%%%%%%%%%%%%%%%%%%%%%%%%%%%%%%%%%%%%%%%%%%%%%%%%%%%%%%%%%%%%%%%%%%%%%

\frame{\frametitle{Why Use the \textit{Beamer} Class?}
\vspace{-.5cm}
\footnotesize
\underline{\textsc{Pros}}
\begin{enumerate}\scriptsize
	\item More bells \& whistles than the \textit{Prosper} class	    
	\item Directly supported by \textcolor{darkgreen}{\textbf{pdflatex}} 
	\begin{itemize}\scriptsize
		\item can still use latex2e, dvips, ps2pdf (\underline{HAVE} to when using \textit{\textcolor{darkgreen}{\textbf{pstricks}}})
	\end{itemize}
	\item Rich overlay \& transition effects
	\item Navigational bars \& symbols
	\item Outputs: screen, handouts, notes, etc.
	\item Customizable \\
\end{enumerate}
\vspace{.5cm}
\underline{\textsc{Cons}}
\begin{enumerate}\scriptsize
	\item Isn't ``what you see is what you get''
\end{enumerate}
}

%%%%%%%%%%%%%%%%%%%%%%%%%%%%%%%%%%%%%%%%%%%%%%%%%%%%%%%%%%%%%%%%%%%%%%%%%%%%%%%%%%%%%%%%%%
%%%%%%%%%%%%%%%%%%%%%%%%%%%%%%%%%%%%%%%% Code %%%%%%%%%%%%%%%%%%%%%%%%%%%%%%%%%%%%%%%%%%%%
%%%%%%%%%%%%%%%%%%%%%%%%%%%%%%%%%%%%%%%%%%%%%%%%%%%%%%%%%%%%%%%%%%%%%%%%%%%%%%%%%%%%%%%%%%

\section{Code}
\subsection{Basic Code}

%%%%%%%%%%%%%%%%%%%%%%%%%%%%%%%%%%%%%%%%%%%%%%%%%%%%%%%%%%%%%%%%%%%%%%%%%%%%%%%%%%%%%%%%%%%

\frame{\frametitle{Basic Code}
\vspace{-.5cm}
\scriptsize
\begin{itemize}
\item Beamer class loading with themes\\
\tiny
	\begin{tabular}{|ll|}
	\hline
	&\\
	\texttt{$\backslash$documentclass\{\textcolor{Blue}{beamer}\}}&\\
	$\backslash$\texttt{mode<presentation>}&\\
	$\backslash$\texttt{usetheme}\{\textcolor{Blue}{\texttt{Warsaw}}\}      &    \texttt{\% Beamer Theme}\\
	$\backslash$\texttt{usecolortheme}\{\textcolor{Blue}{\texttt{lily}}\}    &   \texttt{\% Beamer Color Theme}\\
	&\\\hline 
	\end{tabular}
\vspace{.75cm}
\item \scriptsize Title Page \\
\tiny
	\begin{tabular}{|ll|}
	\hline
	&\\
	$\backslash$\texttt{title}\{\} &\\
	$\backslash$\texttt{subtitle}\{\} &\\
	$\backslash$\texttt{author}\{\}&\\
	$\backslash$\texttt{institute}\{\}&\\
	$\backslash$\texttt{date}\{\}&\\
	&\\
	$\backslash$\texttt{begin}\{\texttt{document}\}&\\
	$\backslash$\texttt{frame}\{ & \texttt{\% the title page}\\
	\hspace{.35cm}$\backslash$\textcolor{Blue}{\texttt{titlepage}}&\\
         \}&\\
	$\vdots$&\\
	$\backslash$\texttt{end}\{\texttt{document}\}&\\
	&\\\hline
	\end{tabular}	
\end{itemize}
}

%%%%%%%%%%%%%%%%%%%%%%%%%%%%%%%%%%%%%%%%%%%%%%%%%%%%%%%%%%%%%%%%%%%%%%%%%%%%%%%%%%%%%%%%%%%

\frame{\frametitle{More Code}
\vspace{-.5cm}
\begin{itemize}
\item \scriptsize Slides\\ \tiny
	\begin{tabular}{|ll|}
	\hline
	&\\
	$\backslash$\texttt{section}\{\} & \\
	$\backslash$\texttt{subsection}\{\} &\\
	$\backslash$\textcolor{Blue}{\texttt{frame}}[\textit{\texttt{options}}]\{ &\\
	$\backslash$\textcolor{Blue}{\texttt{frametitle}}\{\} &\\
	\hspace{.35cm}$\dots$\textit{\texttt{slide contents}}$\dots$ & \\
	\} &\\ 
	&\\\hline
	\end{tabular}
\vspace{.75cm}
\item \scriptsize Many features you want to use require you to load packages, such as:\\ \tiny
\tiny
	\begin{tabular}{|ll|}
	\hline
	&\\
	\texttt{$\backslash$usepackage}\{\textcolor{Blue}{amsmath}\}      &    \texttt{\% for math AMS fonts}\\
	\texttt{$\backslash$usepackage}\{\textcolor{Blue}{graphicx}\}      &    \texttt{\% to include figures}\\
\texttt{$\backslash$usepackage}\{\textcolor{Blue}{subfigure}\}      &    \texttt{\% to have figures in figures}\\
\texttt{$\backslash$usepackage}\{\textcolor{Blue}{multimedia}\}      &    \texttt{\% to include movies}\\
&\\\hline 
\end{tabular}
\end{itemize}
}

%%%%%%%%%%%%%%%%%%%%%%%%%%%%%%%%%%%%%%%%%%%%%%%%%%%%%%%%%%%%%%%%%%%%%%%%%%%%%%%%%%%%%%%%%%%

\subsection{Layout}

%%%%%%%%%%%%%%%%%%%%%%%%%%%%%%%%%%%%%%%%%%%%%%%%%%%%%%%%%%%%%%%%%%%%%%%%%%%%%%%%%%%%%%%%%%%

\frame{\frametitle{Themes}
\footnotesize
\underline{\textsc{Five Theme Categories}}
\begin{enumerate}\scriptsize
	\item \textcolor{darkgreen}{\textbf{Presentation}} (\textit{the slide template})
	\item \textcolor{darkgreen}{\textbf{Color}}$^*$ (\textit{color scheme for slide template})
	\item \textcolor{darkgreen}{\textbf{Font}}$^*$  
	\item \textcolor{darkgreen}{\textbf{Inner}}$^*$ (how you want bullets, boxes, etc. to look)
	\item \textcolor{darkgreen}{\textbf{Outer}}$^*$ (how you want the top/bottom of frames to look)\\
\end{enumerate}
\vspace{.25cm}
\textbf{$^*$ if you don't like the default of the Presentation Theme}\\
\vspace{.5cm}
\underline{\textsc{Example}}\\
\tiny
	\begin{tabular}{|ll|}
	\hline
	&\\
	\texttt{$\backslash$documentclass[compress, red]\{\textcolor{Blue}{beamer}\}}&\\
	\texttt{$\backslash$usetheme}\{\textcolor{Blue}{Warsaw}\}      &    \texttt{\% Beamer Theme}\\
	\texttt{$\backslash$usecolortheme}\{\textcolor{Blue}{lily}\}    &   \texttt{\% Beamer Color Theme}\\
	\texttt{$\backslash$useoutertheme[\textcolor{Blue}{subsection=false}]\{\textcolor{Blue}{smoothbars}\}}& \texttt{\% Beamer Outer Theme}\\
	\texttt{$\backslash$useinnertheme\{\textcolor{Blue}{rectangles}\}}&\texttt{\% Beamer Inner Theme}\\
	&\\\hline 
	\end{tabular}
}

%%%%%%%%%%%%%%%%%%%%%%%%%%%%%%%%%%%%%%%%%%%%%%%%%%%%%%%%%%%%%%%%%%%%%%%%%%%%%%%%%%%%%%%%%%%

\frame{\frametitle{Beamer \textit{Options} Examples}
\begin{itemize}
	\item \texttt{[\textcolor{darkgreen}{\textbf{compress}}]}: makes all navigation bars as small as possible\\
	\textsc{default}: uncompressed\\\vspace{.25cm}
	
	\item  \texttt{[\textcolor{Red}{\textbf{red}}]}: changes color scheme to red\\
	\textsc{default} for beamer theme Warsaw: blue\\\vspace{.25cm}
	
	\item  \texttt{[\textcolor{brown}{\textbf{subsection=false}}]}: removes an extra bar above slide title stating the subsection title\\
	\textsc{default}: true
\end{itemize}
}

%%%%%%%%%%%%%%%%%%%%%%%%%%%%%%%%%%%%%%%%%%%%%%%%%%%%%%%%%%%%%%%%%%%%%%%%%%%%%%%%%%%%%%%%%%%

\frame{\frametitle{Using Color}
\footnotesize
\colorlet{newred}{red!60!black}
\vspace{-1cm}
\begin{center}
 * \hspace{.2cm} \textsc{\textcolor{Blue}{Beamer automatically loads}} \textbf{`xcolor'} \hspace{.2cm} *\\
\end{center}
\begin{itemize} \footnotesize
	\item \underline{Predefined colors}:\\ \vspace{.2cm} 
		\textcolor{red}{red}, \textcolor{blue}{blue}, \textcolor{green}{green}, \textcolor{cyan}{cyan}, \textcolor{magenta}{magenta},
		\textcolor{yellow}{yellow}, \textcolor{black}{black}, \textcolor{darkgray}{darkgray}, \textcolor{gray}{gray},
		\textcolor{lightgray}{lightgray}, \textcolor{orange}{orange}, \textcolor{violet}{violet}, \textcolor{purple}{purple}, \&
		\textcolor{brown}{brown}\\ \vspace{.25cm}
	\item \underline{To define new colors}:\\ \vspace{.2cm} \scriptsize
	\texttt{$\backslash$xdefinecolor\{darkgreen\}\{rgb\}\{0,0.35,0\}}: \textcolor{darkgreen1}{my new color is dark green}\\ \vspace{.15cm}
	\texttt{$\backslash$xdefinecolor\{purpleish\}\{cmyk\}\{0.75,0.75,0,0\}}: \textcolor{purpleish}{color is purple-ish}\\ \vspace{.25cm}
	
	\item \footnotesize \underline{Or substitute colors}:\\ \vspace{.2cm} \scriptsize
	\texttt{$\backslash$colorlet\{newred\}\{red!60!black\}}: \textcolor{newred}{my new color is dark red}
	
\end{itemize}
}

%%%%%%%%%%%%%%%%%%%%%%%%%%%%%%%%%%%%%%%%%%%%%%%%%%%%%%%%%%%%%%%%%%%%%%%%%%%%%%%%%%%%%%%%%%
%%%%%%%%%%%%%%%%%%%%%%%%%%%%%%%% Beamer Features %%%%%%%%%%%%%%%%%%%%%%%%%%%%%%%%%%%%%%%%%
%%%%%%%%%%%%%%%%%%%%%%%%%%%%%%%%%%%%%%%%%%%%%%%%%%%%%%%%%%%%%%%%%%%%%%%%%%%%%%%%%%%%%%%%%%

\section{Beamer Features}
\subsection{Cool Stuff}

%%%%%%%%%%%%%%%%%%%%%%%%%%%%%%%%%%%%%%%%%%%%%%%%%%%%%%%%%%%%%%%%%%%%%%%%%%%%%%%%%%%%%%%%%%

\frame{\frametitle{Overlays}
\vspace{-.5cm}
\footnotesize
There are multiple ways to do overlays:
\begin{enumerate}
	\item \texttt{$\backslash$\textcolor{Blue}{pause}}\\ \vspace{.1cm}
			does the overlay sequentially \\ \vspace{.25cm}
\end{enumerate}

\underline{\textsc{Example}}\\
\pause
\begin{columns}
\column{0.4\textwidth}
\scriptsize
\begin{itemize}
\item I'm 
	\item showing\pause
	\item you
	\item \textit{pause}\pause
\end{itemize}
\hfill
\column{0.6\textwidth}
\tiny
	\begin{tabular}{|ll|}
	\hline
	&\\
	\texttt{$\backslash$begin\{itemize\}}&\\
	    \hspace{.25cm}     \texttt{$\backslash$item I'm} &\\
	    \hspace{.25cm}	\texttt{$\backslash$item showing $\backslash$\textcolor{Blue}{pause}}&\\
	    \hspace{.25cm}	\texttt{$\backslash$item you }&\\
	    \hspace{.25cm}	\texttt{$\backslash$item $\backslash$textit\{pause\} $\backslash$\textcolor{Blue}{pause}}&\\
	\texttt{$\backslash$end\{itemize\}}&\\
	&\\\hline 
	\end{tabular}
\end{columns}

}

%%%%%%%%%%%%%%%%%%%%%%%%%%%%%%%%%%%%%%%%%%%%%%%%%%%%%%%%%%%%%%%%%%%%%%%%%%%%%%%%%%%%%%%%%%

\frame{\frametitle{Overlays}
\vspace{-.5cm}
\footnotesize
There are multiple ways to do overlays:

\begin{enumerate}\footnotesize
	\item \scriptsize \texttt{$\backslash$\textcolor{Blue}{pause}}\\ \vspace{.1cm}
	\item  \footnotesize \texttt{$\backslash$\textcolor{Blue}{item$<$n-$>$}} (means ``from overlay n'')\\ \vspace{.1cm}
	  \texttt{$\backslash$item$<$2$>$} (means ``only overlay 2'')\\ \vspace{.1cm}
	    \texttt{$\backslash$item$<$2,4$>$} (means ``only overlay 2 \& 4'')\\ \vspace{.1cm}
	does non-sequential overlays in the bullet-type (ie. \textcolor{darkgreen}{\texttt{itemize}}),  environments\\ \vspace{.25cm}
\end{enumerate}
\onslide<1->
\underline{\textsc{Example}}\\

\scriptsize
\begin{columns}
\column{0.4\textwidth}
\begin{itemize}
	\item<1> I'm 
	\item<1,2,3-> showing
	\item<2> you
	\item <3->\textit{$\backslash$item$<>$}
\end{itemize}
\hfill
\column{0.6\textwidth}
\onslide<4->
\tiny
	\begin{tabular}{|ll|}
	\hline
	&\\
	\texttt{$\backslash$begin\{itemize\}}&\\
	    \hspace{.25cm}      \texttt{$\backslash$item\textcolor{Blue}{$<$1$>$} I'm} &\\
	    \hspace{.25cm}	\texttt{$\backslash$item\textcolor{Blue}{$<$1,2,3-$>$} showing}&\\
	    \hspace{.25cm}	\texttt{$\backslash$item\textcolor{Blue}{$<$2$>$} you}&\\
	    \hspace{.25cm}	\texttt{$\backslash$item\textcolor{Blue}{$<$3-$>$} $\backslash$textit\{\$$\backslash$backslash\$ item\$$<>$\$\}}&\\
	\texttt{$\backslash$end\{itemize\}}&\\
	&\\\hline 
	\end{tabular}
\end{columns}
}

%%%%%%%%%%%%%%%%%%%%%%%%%%%%%%%%%%%%%%%%%%%%%%%%%%%%%%%%%%%%%%%%%%%%%%%%%%%%%%%%%%%%%%%%%%

\frame{\frametitle{Overlays}
\vspace{-.5cm}
\footnotesize
There are multiple ways to do overlays:

\begin{enumerate}\footnotesize
	\item \scriptsize \texttt{$\backslash$\textcolor{Blue}{pause}}\\ \vspace{.1cm}	
	\item \scriptsize \texttt{$\backslash$\textcolor{Blue}{item}$<$n-$>$}\\ \vspace{.1cm}	
	\item \footnotesize\texttt{$\backslash$\textcolor{Blue}{onslide}$<$n-$>$}\\ \vspace{.1cm}	
	 non-sequential overlays in any environment! \\ \vspace{.1cm}	
\end{enumerate}

\underline{\textsc{Example}}\\

\scriptsize
\begin{columns}
\column{0.35\textwidth}
\begin{itemize}
	\item I'm \onslide<2>showing 
	\item \onslide<3-> showing \onslide<3>you
	\item \onslide<4->you
	\item \textit{$\backslash$onslide$<>$}
\end{itemize}
\hfill
\column{0.65\textwidth}
\onslide<5->
\tiny
	\begin{tabular}{|ll|}
	\hline
	&\\
	\texttt{$\backslash$begin\{itemize\}}&\\
	    \hspace{.25cm}     \texttt{$\backslash$item I'm \textcolor{Blue}{$\backslash$onslide$<$2$>$} showing} &\\
	    \hspace{.25cm}	\texttt{$\backslash$item \textcolor{Blue}{$\backslash$onslide$<$3-$>$} showing \textcolor{Blue}{$\backslash$onslide$<$3$>$} you}&\\
	    \hspace{.25cm}	\texttt{$\backslash$item \textcolor{Blue}{$\backslash$onslide$<$4-$>$} you}&\\
	    \hspace{.25cm}	\texttt{$\backslash$item $\backslash$textit\{\$$\backslash$backslash\$ onslide\$$<>$\$\}}&\\
	\texttt{$\backslash$end\{itemize\}}&\\
	&\\\hline 
	\end{tabular}
\end{columns}
}

%%%%%%%%%%%%%%%%%%%%%%%%%%%%%%%%%%%%%%%%%%%%%%%%%%%%%%%%%%%%%%%%%%%%%%%%%%%%%%%%%%%%%%%%%%

\frame{\frametitle{Overlays}
\vspace{-.5cm}
\footnotesize
There are multiple ways to do overlays:

\begin{enumerate}\footnotesize
	\item \texttt{$\backslash$\textcolor{Blue}{pause}}\\ \vspace{.1cm}	
	\item  \texttt{$\backslash$\textcolor{Blue}{item}$<$n-$>$}\\ \vspace{.1cm}	
	\item \texttt{$\backslash$\textcolor{Blue}{onslide}$<$n-$>$}\\ \vspace{.1cm}
	\item Replace
	\begin{itemize}\scriptsize
		\item \texttt{$\backslash$\textcolor{Blue}{only}$<$n$>$\{$\dots$\}}: successive
		\item \texttt{$\backslash$\textcolor{Blue}{uncover}$<$n$>$\{$\dots$\}}: shows at n
		\item \texttt{$\backslash$\textcolor{Blue}{invisible}$<$n$>$\{$\dots$\}}: hides at n
		\item \texttt{$\backslash$\textcolor{Blue}{alt}$<$n$>$\{at n\}\{not at n\}}: 2 alternatives
		\item \texttt{$\backslash$\textcolor{Blue}{temporal}$<$n$>$\{before\}\{at n\}\{after\}}: 3 alternatives
		\item \texttt{\textcolor{darkgreen}{overprint}} \& \texttt{\textcolor{darkgreen}{overlayarea}} environments
	\end{itemize}
	\item Highlighting
	\begin{itemize}\scriptsize
		\item<+-|alert@+> \texttt{$\backslash$item$<$+-|\textcolor{Blue}{alert}\@+$>$}
		\item<+-|alert@+> \texttt{$\backslash$item$<$2-$>$$\backslash$\textcolor{Blue}{alert}$<$n$>$\{stuff\}}
		\item<+-|alert@+> \texttt{$\backslash$item$<$2-$>$$\backslash$\textcolor{Blue}{alt}$<$3$>$\{$\backslash$color\{green\} stuff\}\{$\backslash$color\{red\} stuff\}}
		\item[]<+-|alert@+>
	\end{itemize} 
\end{enumerate}

}

%%%%%%%%%%%%%%%%%%%%%%%%%%%%%%%%%%%%%%%%%%%%%%%%%%%%%%%%%%%%%%%%%%%%%%%%%%%%%%%%%%%%%%%%%%

\frame{\frametitle{Transition Effects}
\footnotesize
\vspace{-.5cm}
\setbeamercovered{transparent}
\begin{center}
	*\hspace{.15cm}\textsc{This slide uses transparent overlays:}\hspace{.15cm}* \texttt{$\backslash$\textcolor{Blue}{setbeamercovered}\{transparent\}}\\\vspace{.25cm}
\end{center}
\onslide<1->
\underline{Text Animation:}
\onslide<2->
\begin{itemize}
	\item \texttt{$\backslash$\textcolor{Blue}{animate}}, \texttt{$\backslash$\textcolor{Blue}{animatevalue}}, etc.
	\item can do timed overlays, etc.\\\vspace{.2cm}
\end{itemize}
\onslide<1->
\underline{Slide Transitions:}\\
\onslide<3->
\begin{itemize}\scriptsize
	\item Seven options: Blinds, Box, Dissolve, Glitter, Replace, Split, Wipe\\\vspace{.15cm}
	\textsc{Examples}
	\begin{itemize} \scriptsize
		\item Dissolve:\texttt{$\backslash$\color{Blue}transdissolve}
		\item Glitter: \texttt{$\backslash$\color{Blue}transglitter[direction=90]}
		\item Split (2 vertical lines sweep outward): \texttt{$\backslash$\color{Blue}transsplitverticalout}
	\end{itemize}
\end{itemize}
}

%%%%%%%%%%%%%%%%%%%%%%%%%%%%%%%%%%%%%%%%%%%%%%%%%%%%%%%%%%%%%%%%%%%%%%%%%%%%%%%%%%%%%%%%%%

\frame{\frametitle{Figures}
\transdissolve
\begin{itemize}
	\item Standard \LaTeX  \hspace{.05cm} \texttt{\textcolor{darkgreen} {figure}} environment can be used.\\\vspace{.1cm}
	\item Using the \textbf{`graphicx'} package:\\
	\begin{itemize}
		\item doesn't support all figures types: \\ \vspace{.25cm}
		easy fix: make ALL figures pdfs \\(eg. convert eps using \textcolor{darkgreen}{`epstopdf'}) \\ \vspace{.15cm}
		\vspace{.1cm}
	\tiny
	\begin{tabular}{|ll|}
	\hline
	&\\
	\texttt{$\backslash$begin\{figure\}} &\\
	\hspace{.25cm} \texttt{$\backslash$\textcolor{Blue}{includegraphics}[width=$\backslash$columnwidth]\{myprettyfigure\}} &\\
	\texttt{$\backslash$end\{figure\}} &\\
	&\\\hline 
	\end{tabular}
	\end{itemize}\vspace{.1cm}
	
\item can also use \texttt{$\backslash$\textcolor{Blue} {pgfimage}}\\
	\tiny
	\begin{tabular}{|ll|}
	\hline
	&\\
	\texttt{$\backslash$\textcolor{Blue}{pgfimage}[height=4cm]\{myprettyfigure\}} &\\
	&\\\hline 
	\end{tabular}
\end{itemize}
\footnotesize
\begin{center}
*  \hspace{.25cm} \textbf{NOTICE that you don't have to specify the file type} \hspace{.25cm} *
\end{center}
}

%%%%%%%%%%%%%%%%%%%%%%%%%%%%%%%%%%%%%%%%%%%%%%%%%%%%%%%%%%%%%%%%%%%%%%%%%%%%%%%%%%%%%%%%%%

\frame{\frametitle{Figures - Zooming}
\vspace{-.5cm}
\footnotesize
\begin{itemize}\scriptsize
	\item You can zoom into portions of your figures\\\vspace{.1cm}
	\tiny
	\begin{tabular}{|ll|}
	\hline
	&\\
	\texttt{$\backslash$\textcolor{Blue}{framezoom}$<1>$$<2>$[border](0cm, 3.5cm)(2.75cm, 1cm)} &\\
	\texttt{$\backslash$\textcolor{Blue}{framezoom}$<1>$$<3>$[border](3cm, 3.5cm)(1cm, 1cm)} &\\
	\texttt{$\backslash$pgfimage[height=4cm]\{ambersmice\}} &\\
	&\\\hline 
	\end{tabular}
\end{itemize}
\framezoom<1><2>[border](0cm,3.5cm)(2.75cm, 1cm)
\framezoom<1><3>[border](3cm,3.5cm)(1cm, 1cm)
\pgfimage[height=4cm]{ambersmice}
}

%%%%%%%%%%%%%%%%%%%%%%%%%%%%%%%%%%%%%%%%%%%%%%%%%%%%%%%%%%%%%%%%%%%%%%%%%%%%%%%%%%%%%%%%%%

\frame{\frametitle{Movies}
\vspace{-.5cm}
\scriptsize
\begin{center}
	\movie[height=1.125in,width=1.5in,poster]{}{Chemotaxis.mov}
\end{center}

	\tiny
	\begin{tabular}{|ll|}
	\hline
	&\\
	\texttt{$\backslash$usepackage\{\textcolor{Blue}{multimedia}\}} &\\
	$\vdots$ &\\
	\texttt{$\backslash$frame\{} &\\
	\hspace{.25cm}\texttt{$\backslash$\textcolor{Blue}{movie}[height=1.125in,width=1.5in,poster]\{\}\{Chemotaxis.mov\}} &\\
	\} &\\
	&\\\hline 
	\end{tabular}
\scriptsize
\vspace{.15cm}
\begin{itemize}\scriptsize
	\item[*] \tiny \texttt{$\backslash$\textcolor{Blue}{movie}[\textit{options}]\{\textit{text, picture, etc to click on}\}\{\textit{name of movie}\}}
	\item[*] \scriptsize Should support all major movie types: \textcolor{darkgreen}{.avi}, \textcolor{darkgreen}{.mov}, etc.\\\vspace{.1cm}
	\textbf{Problems: make sure Acrobat has the correct plug-ins!!! \\
	Does NOT work on Linux/Unix systems?!?!}\\
	\item[*] You may need to use the \texttt{\textcolor{darkgreen}{externalviewer}} option
\end{itemize}
}

%%%%%%%%%%%%%%%%%%%%%%%%%%%%%%%%%%%%%%%%%%%%%%%%%%%%%%%%%%%%%%%%%%%%%%%%%%%%%%%%%%%%%%%%%%

\frame{\frametitle{Using Columns}
\vspace{-.5cm}
\footnotesize
The \texttt{\textcolor{darkgreen}{column}} environment is \textbf{extremely useful}!\\\vspace{.1cm}
\begin{itemize}
	\item allows you to add as many columns as you want\\\vspace{.1cm}
	\item can put multiple column environments on any page\\ \vspace{.2cm}
\tiny
\begin{tabular}{|ll|}
\hline
&\\
\texttt{$\backslash$begin\{columns\}[t]} &\\
\hspace{.25cm}\texttt{$\backslash$\textcolor{Blue}{column}\{0.25$\backslash$textwidth\}} &\\
\hspace{.5cm} $\dots$ \textit{contents} $\dots$ &\\
\hspace{.25cm}\texttt{$\backslash$\textcolor{Blue}{column}\{0.5$\backslash$textwidth\}} &\\
\hspace{.5cm} $\dots$ \textit{contents} $\dots$ &\\
\hspace{.25cm}\texttt{$\backslash$\textcolor{Blue}{column}\{0.25$\backslash$textwidth\}} &\\
\hspace{.5cm} $\dots$ \textit{contents} $\dots$ &\\
\texttt{$\backslash$end\{columns\}} &\\
&\\\hline 
\end{tabular}
\end{itemize}
}

%%%%%%%%%%%%%%%%%%%%%%%%%%%%%%%%%%%%%%%%%%%%%%%%%%%%%%%%%%%%%%%%%%%%%%%%%%%%%%%%%%%%%%%%%%

\frame{\frametitle{Theorems, etc.}
\scriptsize
The \texttt{\textcolor{darkgreen}{theorem}} , \texttt{\textcolor{darkgreen}{proof}} , \texttt{\textcolor{darkgreen}{block}}, \texttt{\textcolor{darkgreen}{example}}, \texttt{\textcolor{darkgreen}{definition}}, etc. environments:
\begin{itemize}
	\item For theorems/proofs\\
	\begin{theorem}\label{theorem1}
		Write your fantastic \\
		theorem here $\dots$
	\end{theorem}
\tiny
	\begin{tabular}{|ll|}
	\hline
	&\\
	\texttt{$\backslash$begin\{\textcolor{Blue}{theorem}\}} &\\
	\hspace{.25cm}Write your fantastic $\backslash$$\backslash$&\\
	\hspace{.25cm}theorem here \$$\backslash$dots\$&\\
	\texttt{$\backslash$end\{\textcolor{Blue}{theorem}\}} &\\
	&\\\hline 
	\end{tabular}
\vspace{0.5cm}
	\item \scriptsize Or to highlight points:\\
	\begin{block}{Summary}
		\begin{itemize}
			\item Beamer is cool!
		\end{itemize}
	\end{block}
\end{itemize}

\tiny
	\begin{tabular}{|ll|}
	\hline
	&\\
	\texttt{$\backslash$begin\{\textcolor{Blue}{block}\}\{Summary\}} &\\
	\hspace{.25cm}$\backslash$begin\{itemize\} &\\
	\hspace{.25cm}$\backslash$item Beamer is cool! &\\
	\hspace{.25cm}$\backslash$end\{itemize\} &\\
	\texttt{$\backslash$end\{\textcolor{Blue}{block}\}} &\\
	&\\\hline 
	\end{tabular}
}

%%%%%%%%%%%%%%%%%%%%%%%%%%%%%%%%%%%%%%%%%%%%%%%%%%%%%%%%%%%%%%%%%%%%%%%%%%%%%%%%%%%%%%%%%%

\section{More \LaTeX}
\subsection{Ahh..sweet}

%%%%%%%%%%%%%%%%%%%%%%%%%%%%%%%%%%%%%%%%%%%%%%%%%%%%%%%%%%%%%%%%%%%%%%%%%%%%%%%%%%%%%%%%%%

\frame[label=MyVerbatim]{\frametitle{Fragile Environments \& Hyperlinks}
\vspace{-.4cm}
\footnotesize
\underline{Fragile Environments}\\\vspace{.1cm}
You \underline{CANNOT} use \texttt{\textcolor{darkgreen}{verbatim}} without specifying it in the frame \textit{options}:\\ \vspace{.2cm}
\tiny
	\begin{tabular}{|ll|}
	\hline
	&\\
	\texttt{$\backslash$frame[\textcolor{Blue}{containsverbatim}]\{ $\backslash$frametitle\{\} } &\\
	\hspace{.25cm}\texttt{$\backslash$begin\{verbatim\}} &\\
	\hspace{.5cm} $\dots$ \textit{contents} $\dots$&\\
	\hspace{.25cm}\texttt{$\backslash$end\{verbatim\}} &\\
	\} &\\
	&\\\hline 
	\end{tabular}
\newline
\newline
\footnotesize
\onslide<2->
\underline{Hyperlinks \& Buttons:} \\ \vspace{.1cm}
You can create \textcolor{darkgreen}{buttons} to jump around your talk:
	\hyperlink{theorem1}{\beamergotobutton{Jump to Theorem \#1}}
	\hypertarget{theorem1}{}
\vspace{.15cm}
\begin{itemize}
	\item You need to put a \textcolor{darkgreen}{label} on the slide: \scriptsize\texttt{$\backslash$frame[label=MyVerbatim]\{}\\
	OR, \texttt{$\backslash$label\{theorem1\}} \\\vspace{.1cm}
	\item \footnotesize To create the button:\\\vspace{.1cm}
\tiny
	\begin{tabular}{|ll|}
	\hline
	&\\
	\texttt{$\backslash$usepackage\{\textcolor{Blue}{hyperref}\}} &\\
	\texttt{$\backslash$frame\{} &\\
	\texttt{$\backslash$\textcolor{Blue}{hyperlink}\{theorem1\}\{$\backslash$\textcolor{Blue}{beamergotobutton}\{Jump to Theorem $\backslash$\#1\}\}} &\\
	\texttt{$\backslash$\textcolor{Blue}{hypertarget}\{theorem1\}\{\}} &\\
	\}&\\
	&\\\hline 
	\end{tabular}
\end{itemize}
}

%%%%%%%%%%%%%%%%%%%%%%%%%%%%%%%%%%%%%%%%%%%%%%%%%%%%%%%%%%%%%%%%%%%%%%%%%%%%%%%%%%%%%%%%%%

\frame{\frametitle{And, Finally $\dots$} 
\underline{\textsc{Other useful things:}}\\
\footnotesize
\begin{itemize}
	\item Drawing diagrams\\
	\begin{itemize}\footnotesize
		\item[*] \textcolor{darkgreen}{xypic}: draws the diagrams I showed at beginning\\\vspace{.2cm}
		\item[*] the \LaTeX  \hspace{.05cm} \texttt{\textcolor{darkgreen}{picture}} environment \\\vspace{.2cm}
		\item[*] \texttt{\textcolor{darkgreen}{pstricks}}: can't use \textbf{pdflatex} with this\\ \vspace{.2cm}
	\end{itemize}
\item Logo in the footer:\\
\begin{itemize}
	\item[*] put \texttt{$\backslash$\textcolor{Blue}{logo}\{name\}} in preamble\\\vspace{.2cm}
	\item[*] puts logo in bottom right corner\\ \vspace{.2cm}
\end{itemize}
\item References\\
\begin{itemize}
	\item[*] Beamer Users Guide:\\
	 \tiny \url{www.ctan.org/tex-archive/macros/latex/contrib/beamer/doc/beameruserguide.pdf}\\ \vspace{.2cm}\footnotesize
	 \item[*] \footnotesize Google: if you think Beamer should be able to do it, Google it.
\end{itemize}
\end{itemize}
}

%%%%%%%%%%%%%%%%%%%%%%%%%%%%%%%%%%%%%%%%%%%%%%%%%%%%%%%%%%%%%%%%%%%%%%%%%%%%%%%%%%%%%%%%%%
%%%%%%%%%%%%%%%%%%%%%%%%%%%%%% End Your Document %%%%%%%%%%%%%%%%%%%%%%%%%%%%%%%%%%%%%%%%%
%%%%%%%%%%%%%%%%%%%%%%%%%%%%%%%%%%%%%%%%%%%%%%%%%%%%%%%%%%%%%%%%%%%%%%%%%%%%%%%%%%%%%%%%%%

\end{document}

